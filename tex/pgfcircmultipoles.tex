% Copyright 2007-2009 by Massimo Redaelli
% Copyright 2019 by Romano Giannetti
%
% This file may be distributed and/or modified
%
% 1. under the LaTeX Project Public License and/or
% 2. under the GNU Public License.
%
% See the files gpl-3.0_license.txt and lppl-1-3c_license.txt for more details.

%%%%%%%%%%%%%%%%%%%%%%%%%%%%%%%%%%%%%%
%% Multipoles
%%%%%%%%%%%%%%%%%%%%%%%%%%%%%%%%%%%%%%

% counters for pin accounting...
\newcount\pgf@circ@count@c
\newcount\pgf@circ@count@b
\def\pgf@circ@dip@pin@shift{0.5}
\def\pgf@circ@qfp@pin@shift{0.25}

% derived from https://tex.stackexchange.com/a/146753/38080
% original author Mark Wibrow
% Thanks also to John Kormylo https://tex.stackexchange.com/a/372996/38080
% a lot of thanks to @marmot  for the un-rotation hint
% https://tex.stackexchange.com/a/473571/38080

% DIP (dual in line package) chips

\pgfdeclareshape{dipchip}{
    \savedmacro\numpins{%
            \c@pgf@counta=\pgfkeysvalueof{/tikz/circuitikz/multipoles/dipchip/num pins}%
            \def\numpins{\the\c@pgf@counta}
    }
    \savedanchor\centerpoint{%
        \pgf@x=-.5\wd\pgfnodeparttextbox%
        \pgf@y=-.5\ht\pgfnodeparttextbox%
        \advance\pgf@y by+.5\dp\pgfnodeparttextbox%
    }%
    \savedanchor\origin{\pgfpoint{0pt}{0pt}}
    \anchor{center}{\origin}
    \anchor{text}{\centerpoint}% to adjust text
    \saveddimen\height{%
        \pgfmathsetlength\pgf@x{((\numpins)
        *\pgfkeysvalueof{/tikz/circuitikz/multipoles/dipchip/pin spacing})*\pgf@circ@Rlen/2}%
    }%
    \saveddimen{\chipspacing}{\pgfmathsetlength\pgf@x{\pgf@circ@Rlen*\pgfkeysvalueof{/tikz/circuitikz/multipoles/dipchip/pin spacing}}}
    \saveddimen{\width}{\pgfmathsetlength\pgf@x{\pgf@circ@Rlen*\pgfkeysvalueof{/tikz/circuitikz/multipoles/dipchip/width}}}
    \saveddimen{\extshift}{\pgfmathsetlength\pgf@x{\pgf@circ@Rlen*\pgfkeysvalueof{/tikz/circuitikz/multipoles/external pins width}}}
    % standard anchors
    \savedanchor\northwest{%
        \pgfmathsetlength\pgf@y{0.5*((\numpins)
        *\pgfkeysvalueof{/tikz/circuitikz/multipoles/dipchip/pin spacing})*\pgf@circ@Rlen/2}%
        \pgfmathsetlength\pgf@x{-0.5*\pgf@circ@Rlen*\pgfkeysvalueof{/tikz/circuitikz/multipoles/dipchip/width}}
    }
    \anchor{dot}{\northwest
        \pgfmathsetlength\pgf@x{\pgf@x + 0.3*\chipspacing}
        \pgfmathsetlength\pgf@y{\pgf@y - 0.3*\chipspacing}
    }
    \anchor{nw}{\northwest}
    \anchor{ne}{\northwest\pgf@x=-\pgf@x}
    \anchor{se}{\northwest\pgf@x=-\pgf@x\pgf@y=-\pgf@y}
    \anchor{sw}{\northwest\pgf@y=-\pgf@y}
    \anchor{north west}{\northwest}
    \anchor{north east}{\northwest\pgf@x=-\pgf@x}
    \anchor{south east}{\northwest\pgf@x=-\pgf@x \pgf@y=-\pgf@y}
    \anchor{south west}{\northwest\pgf@y=-\pgf@y}
    \anchor{n}{\northwest\pgf@x=0pt }
    \anchor{e}{\northwest\pgf@x=-\pgf@x\pgf@y=0pt }
    \anchor{s}{\northwest\pgf@x=0pt\pgf@y=-\pgf@y}
    \anchor{w}{\northwest\pgf@y=0pt }
    \anchor{north}{\northwest\pgf@x=0pt }
    \anchor{east}{\northwest\pgf@x=-\pgf@x\pgf@y=0pt }
    \anchor{south}{\northwest\pgf@x=0pt\pgf@y=-\pgf@y}
    \anchor{west}{\northwest\pgf@y=0pt }
    % start drawing
    \backgroundpath{%
        \northwest
        \pgf@circ@res@up = \pgf@y
        \pgf@circ@res@down = -\pgf@y
        \pgf@circ@res@right = -\pgf@x
        \pgf@circ@res@left = \pgf@x
        \pgf@circ@res@step = \pgfkeysvalueof{/tikz/circuitikz/multipoles/dipchip/pin spacing}\pgf@circ@Rlen
        \pgf@circ@res@other = \pgfkeysvalueof{/tikz/circuitikz/multipoles/external pins width}\pgf@circ@Rlen
        \pgfscope% (for the line width)
        \pgfsetlinewidth{\pgfkeysvalueof{/tikz/circuitikz/multipoles/thickness}\pgflinewidth}
        \pgfpathrectanglecorners{\pgfpoint{-\width/2}{-\height/2}}{\pgfpoint{\width/2}{\height/2}}%
        \pgf@circ@draworfill
        %% upside mark
        \ifpgf@circuit@chip@topmark
            \pgfpathmoveto{\pgfpoint{0.2*\pgf@circ@res@left}{\pgf@circ@res@up}}
            \pgfpatharc{0}{180}{0.2*\pgf@circ@res@left}
        \fi
        \pgfusepath{stroke}%
        \pgfsetcolor{\pgfkeysvalueof{/tikz/circuitikz/color}}
        % Adding the pin number
        \ifpgf@circuit@chip@shownumbers
            \c@pgf@counta=\numpins\relax
            \divide\c@pgf@counta by 2 \pgf@circ@count@b=\c@pgf@counta
            % thanks to @marmot: https://tex.stackexchange.com/a/473571/38080
            \ifpgf@circuit@chip@straightnumbers
                \pgfgettransformentries\a\b\temp\temp\temp\temp
                \pgfmathsetmacro{\rot}{-atan2(\b,\a)}
                \pgfmathtruncatemacro{\quadrant}{mod(4+int(360+(\rot+45)/90),4)}
            \else
                \pgfmathsetmacro{\rot}{0}
                \pgfmathsetmacro{\quadrant}{0}
            \fi
            \def\pgf@circ@strut{\vrule width 0pt height 1em depth 0.4em\relax}
            \def\mytext{\pgfkeysvalueof{/tikz/circuitikz/multipoles/font}\space\pgf@circ@strut\the\pgf@circ@count@c\space}
            \pgfmathloop%
            \ifnum\c@pgf@counta>0
                \ifcase\quadrant % rotation 0
                    % left
                    \pgf@circ@count@c=\c@pgf@counta
                    \pgftext[left,
                        at=\pgfpoint{\pgf@circ@res@left}{\pgf@circ@res@up+(\pgf@circ@dip@pin@shift-\the\c@pgf@counta)*\pgf@circ@res@step},
                        rotate=\rot]{\mytext}
                    % right
                    \pgf@circ@count@c=\numexpr2*\pgf@circ@count@b-\c@pgf@counta+1\relax
                    \pgftext[right,
                        at=\pgfpoint{\pgf@circ@res@right}{\pgf@circ@res@up+(\pgf@circ@dip@pin@shift-\the\c@pgf@counta)*\pgf@circ@res@step},
                        rotate=\rot]{\mytext}
                \or % rotation -90
                    % left
                    \pgf@circ@count@c=\c@pgf@counta
                    \pgftext[top,
                        at=\pgfpoint{\pgf@circ@res@left}{\pgf@circ@res@up+(\pgf@circ@dip@pin@shift-\the\c@pgf@counta)*\pgf@circ@res@step},
                        rotate=\rot]{\mytext}
                    % right
                    \pgf@circ@count@c=\numexpr2*\pgf@circ@count@b-\c@pgf@counta+1\relax
                    \pgftext[bottom,
                        at=\pgfpoint{\pgf@circ@res@right}{\pgf@circ@res@up+(\pgf@circ@dip@pin@shift-\the\c@pgf@counta)*\pgf@circ@res@step},
                        rotate=\rot]{\mytext}
                \or %rotation 180
                    % left
                    \pgf@circ@count@c=\c@pgf@counta
                    \pgftext[right,
                        at=\pgfpoint{\pgf@circ@res@left}{\pgf@circ@res@up+(\pgf@circ@dip@pin@shift-\the\c@pgf@counta)*\pgf@circ@res@step},
                        rotate=\rot]{\mytext}
                    % right
                    \pgf@circ@count@c=\numexpr2*\pgf@circ@count@b-\c@pgf@counta+1\relax
                    \pgftext[left,
                        at=\pgfpoint{\pgf@circ@res@right}{\pgf@circ@res@up+(\pgf@circ@dip@pin@shift-\the\c@pgf@counta)*\pgf@circ@res@step},
                        rotate=\rot]{\mytext}
                \or % rotation +90
                    % left
                    \pgf@circ@count@c=\c@pgf@counta
                    \pgftext[bottom,
                        at=\pgfpoint{\pgf@circ@res@left}{\pgf@circ@res@up+(\pgf@circ@dip@pin@shift-\the\c@pgf@counta)*\pgf@circ@res@step},
                        rotate=\rot]{\mytext}
                    % right
                    \pgf@circ@count@c=\numexpr2*\pgf@circ@count@b-\c@pgf@counta+1\relax
                    \pgftext[top,
                        at=\pgfpoint{\pgf@circ@res@right}{\pgf@circ@res@up+(\pgf@circ@dip@pin@shift-\the\c@pgf@counta)*\pgf@circ@res@step},
                        rotate=\rot]{\mytext}
                \fi
                \advance\c@pgf@counta-1\relax%
                \repeatpgfmathloop
            \fi
            \endpgfscope
            \ifdim\pgf@circ@res@other>0pt
            \pgfscope
                \pgfsetlinewidth{\pgfkeysvalueof{/tikz/circuitikz/multipoles/external pins thickness}\pgflinewidth}
                \c@pgf@counta=\numpins\relax
                \divide\c@pgf@counta by 2 \pgf@circ@count@b=\c@pgf@counta
                \pgfmathloop%
                \ifnum\c@pgf@counta>0
                    \edef\padfrac{\pgfkeysvalueof{/tikz/circuitikz/multipoles/external pad fraction}}
                    \ifnum\padfrac>0
                        \pgf@circ@res@temp=\pgf@circ@res@step\divide\pgf@circ@res@temp by \padfrac
                        % left side pads
                        \pgfpathmoveto{\pgfpoint{\pgf@circ@res@left}{\pgf@circ@res@temp+\pgf@circ@res@up+(\pgf@circ@dip@pin@shift-\the\c@pgf@counta)*\pgf@circ@res@step}}
                        \pgfpathlineto{\pgfpoint{\pgf@circ@res@left-\pgf@circ@res@other}{\pgf@circ@res@temp+\pgf@circ@res@up+(\pgf@circ@dip@pin@shift-\the\c@pgf@counta)*\pgf@circ@res@step}}
                        \pgfpathlineto{\pgfpoint{\pgf@circ@res@left-\pgf@circ@res@other}{-\pgf@circ@res@temp+\pgf@circ@res@up+(\pgf@circ@dip@pin@shift-\the\c@pgf@counta)*\pgf@circ@res@step}}
                        \pgfpathlineto{\pgfpoint{\pgf@circ@res@left}{-\pgf@circ@res@temp+\pgf@circ@res@up+(\pgf@circ@dip@pin@shift-\the\c@pgf@counta)*\pgf@circ@res@step}}
                        % right side pads
                        \pgfpathmoveto{\pgfpoint{\pgf@circ@res@right}{\pgf@circ@res@temp+\pgf@circ@res@up+(\pgf@circ@dip@pin@shift-\the\c@pgf@counta)*\pgf@circ@res@step}}
                        \pgfpathlineto{\pgfpoint{\pgf@circ@res@right+\pgf@circ@res@other}{\pgf@circ@res@temp+\pgf@circ@res@up+(\pgf@circ@dip@pin@shift-\the\c@pgf@counta)*\pgf@circ@res@step}}
                        \pgfpathlineto{\pgfpoint{\pgf@circ@res@right+\pgf@circ@res@other}{-\pgf@circ@res@temp+\pgf@circ@res@up+(\pgf@circ@dip@pin@shift-\the\c@pgf@counta)*\pgf@circ@res@step}}
                        \pgfpathlineto{\pgfpoint{\pgf@circ@res@right}{-\pgf@circ@res@temp+\pgf@circ@res@up+(\pgf@circ@dip@pin@shift-\the\c@pgf@counta)*\pgf@circ@res@step}}
                    \else
                        % left side pins
                        \pgfpathmoveto{\pgfpoint{\pgf@circ@res@left}{\pgf@circ@res@up+(\pgf@circ@dip@pin@shift-\the\c@pgf@counta)*\pgf@circ@res@step}}
                        \pgfpathlineto{\pgfpoint{\pgf@circ@res@left-\pgf@circ@res@other}{\pgf@circ@res@up+(\pgf@circ@dip@pin@shift-\the\c@pgf@counta)*\pgf@circ@res@step}}
                        % right side pins
                        \pgfpathmoveto{\pgfpoint{\pgf@circ@res@right}{\pgf@circ@res@up+(\pgf@circ@dip@pin@shift-\the\c@pgf@counta)*\pgf@circ@res@step}}
                        \pgfpathlineto{\pgfpoint{\pgf@circ@res@right+\pgf@circ@res@other}{\pgf@circ@res@up+(\pgf@circ@dip@pin@shift-\the\c@pgf@counta)*\pgf@circ@res@step}}
                    \fi
                    \advance\c@pgf@counta by -1\relax%
                \repeatpgfmathloop
                \pgfusepath{stroke}
            \endpgfscope
            \fi
        }%
        % \pgf@sh@s@<name of the shape here> contains all the code for the shape
        % and is executed just before a node is drawn.
        \pgfutil@g@addto@macro\pgf@sh@s@dipchip{%
            % Start with the maximum pin number and go backwards.
            \c@pgf@counta=\numpins\relax
            \pgfmathloop%
            \ifnum\c@pgf@counta>0
                % we will create two anchors per pin: the "normal one" like `pin 1` for the
                % electrical contact, and the "border one" like `bpin 1` for labels.
                % they will coincide if `external pins width` is set to 0.
                \expandafter\xdef\csname pgf@anchor@dipchip@pin\space\the\c@pgf@counta\endcsname{%
                    \noexpand\pgf@circ@dippinanchor{\the\c@pgf@counta}{1}%
                }
                \expandafter\xdef\csname pgf@anchor@dipchip@bpin\space\the\c@pgf@counta\endcsname{%
                    \noexpand\pgf@circ@dippinanchor{\the\c@pgf@counta}{0}%
                }
                \advance\c@pgf@counta by -1\relax%
                \repeatpgfmathloop%
            }%
        }

% QFP (quad flat package) chips

\pgfdeclareshape{qfpchip}{
    \savedmacro\numpins{%
            \c@pgf@counta=\pgfkeysvalueof{/tikz/circuitikz/multipoles/qfpchip/num pins}%
            \def\numpins{\the\c@pgf@counta}
    }
    \savedanchor\centerpoint{%
        \pgf@x=-.5\wd\pgfnodeparttextbox%
        \pgf@y=-.5\ht\pgfnodeparttextbox%
        \advance\pgf@y by+.5\dp\pgfnodeparttextbox%
    }%
    \savedanchor\origin{\pgfpoint{0pt}{0pt}}
    \anchor{center}{\origin}
    \anchor{text}{\centerpoint}% to adjust text
    \saveddimen\height{%
        \pgfmathsetlength\pgf@x{((\numpins+2)
        *\pgfkeysvalueof{/tikz/circuitikz/multipoles/qfpchip/pin spacing})*\pgf@circ@Rlen/4}%
    }%
    \saveddimen\width{%
        \pgfmathsetlength\pgf@x{((\numpins+2)
        *\pgfkeysvalueof{/tikz/circuitikz/multipoles/qfpchip/pin spacing})*\pgf@circ@Rlen/4}%
    }%
    \saveddimen{\chipspacing}{\pgfmathsetlength\pgf@x{\pgf@circ@Rlen*\pgfkeysvalueof{/tikz/circuitikz/multipoles/qfpchip/pin spacing}}}
    \saveddimen{\extshift}{\pgfmathsetlength\pgf@x{\pgf@circ@Rlen*\pgfkeysvalueof{/tikz/circuitikz/multipoles/external pins width}}}
    % standard anchors
    \savedanchor\northwest{%
        \pgfmathsetlength\pgf@y{0.5*((\numpins+2)
        *\pgfkeysvalueof{/tikz/circuitikz/multipoles/qfpchip/pin spacing})*\pgf@circ@Rlen/4}%
        \pgf@x=-\pgf@y
    }
    \anchor{dot}{\northwest
        \pgfmathsetlength\pgf@x{\pgf@x + 0.3*\chipspacing}
        \pgfmathsetlength\pgf@y{\pgf@y - 0.3*\chipspacing}
    }
    \anchor{nw}{\northwest}
    \anchor{ne}{\northwest\pgf@x=-\pgf@x}
    \anchor{se}{\northwest\pgf@x=-\pgf@x\pgf@y=-\pgf@y}
    \anchor{sw}{\northwest\pgf@y=-\pgf@y}
    \anchor{north west}{\northwest}
    \anchor{north east}{\northwest\pgf@x=-\pgf@x}
    \anchor{south east}{\northwest\pgf@x=-\pgf@x \pgf@y=-\pgf@y}
    \anchor{south west}{\northwest\pgf@y=-\pgf@y}
    \anchor{n}{\northwest\pgf@x=0pt }
    \anchor{e}{\northwest\pgf@x=-\pgf@x\pgf@y=0pt }
    \anchor{s}{\northwest\pgf@x=0pt\pgf@y=-\pgf@y}
    \anchor{w}{\northwest\pgf@y=0pt }
    \anchor{north}{\northwest\pgf@x=0pt }
    \anchor{east}{\northwest\pgf@x=-\pgf@x\pgf@y=0pt }
    \anchor{south}{\northwest\pgf@x=0pt\pgf@y=-\pgf@y}
    \anchor{west}{\northwest\pgf@y=0pt }
    % start drawing
    \backgroundpath{%
        \northwest
        \pgf@circ@res@up = \pgf@y
        \pgf@circ@res@down = -\pgf@y
        \pgf@circ@res@right = -\pgf@x
        \pgf@circ@res@left = \pgf@x
        \pgf@circ@res@step = \pgfkeysvalueof{/tikz/circuitikz/multipoles/qfpchip/pin spacing}\pgf@circ@Rlen
        \pgf@circ@res@other = \pgfkeysvalueof{/tikz/circuitikz/multipoles/external pins width}\pgf@circ@Rlen
        \pgfscope% (for the line width)
        \pgfsetlinewidth{\pgfkeysvalueof{/tikz/circuitikz/multipoles/thickness}\pgflinewidth}
        %% upside mark
        \ifpgf@circuit@chip@topmark
            \pgfpathmoveto{\pgfpoint{-\width/2}{\height/2-\pgf@circ@res@step/2}}
            \pgfpathlineto{\pgfpoint{-\width/2+\pgf@circ@res@step/2}{\height/2}}
        \else
            \pgfpathmoveto{\pgfpoint{-\width/2}{\height/2}}
        \fi
        %% rest of the shape
        \pgfpathlineto{\pgfpoint{\width/2}{\height/2}}
        \pgfpathlineto{\pgfpoint{\width/2}{-\height/2}}
        \pgfpathlineto{\pgfpoint{-\width/2}{-\height/2}}
        \pgfpathclose
        \pgf@circ@draworfill
        % Adding the pin number
        \pgfsetcolor{\pgfkeysvalueof{/tikz/circuitikz/color}}
        \ifpgf@circuit@chip@shownumbers
            \c@pgf@counta=\numpins%
            \divide\c@pgf@counta by 4 \pgf@circ@count@b=\c@pgf@counta
            % thanks to @marmot: https://tex.stackexchange.com/a/473571/38080
            \ifpgf@circuit@chip@straightnumbers
                \pgfgettransformentries\a\b\temp\temp\temp\temp
                \pgfmathsetmacro{\rot}{-atan2(\b,\a)}
                \pgfmathtruncatemacro{\quadrant}{mod(4+int(360+(\rot+45)/90),4)}
            \else
                \pgfmathsetmacro{\rot}{0}
                \pgfmathsetmacro{\quadrant}{0}
            \fi
            \def\pgf@circ@strut{\vrule width 0pt height 1em depth 0.4em\relax}
            \def\mytext{\pgfkeysvalueof{/tikz/circuitikz/multipoles/font}\space\pgf@circ@strut\the\pgf@circ@count@c\space}
            \pgfmathloop%
            \ifnum\c@pgf@counta>0
                \ifcase\quadrant % rotation 0
                    % left
                    \pgf@circ@count@c=\c@pgf@counta
                    \pgftext[left,
                        at=\pgfpoint{\pgf@circ@res@left}{\pgf@circ@res@up+(\pgf@circ@qfp@pin@shift-\the\c@pgf@counta)*\pgf@circ@res@step},
                        rotate=\rot]{\mytext}
                    % bottom
                    \pgf@circ@count@c=\numexpr\pgf@circ@count@b+\c@pgf@counta\relax
                    \pgftext[bottom,
                        at=\pgfpoint{\pgf@circ@res@left-(\pgf@circ@qfp@pin@shift-\the\c@pgf@counta)*\pgf@circ@res@step}{\pgf@circ@res@down},
                        rotate=\rot]{\mytext}
                    % right
                    \pgf@circ@count@c=\numexpr3*\pgf@circ@count@b-\c@pgf@counta+1\relax
                    \pgftext[right,
                        at=\pgfpoint{\pgf@circ@res@right}{\pgf@circ@res@up+(\pgf@circ@qfp@pin@shift-\the\c@pgf@counta)*\pgf@circ@res@step},
                        rotate=\rot]{\mytext}
                    % top
                    \pgf@circ@count@c=\numexpr3*\pgf@circ@count@b+\c@pgf@counta\relax
                    \pgftext[top,
                        at=\pgfpoint{\pgf@circ@res@right+(\pgf@circ@qfp@pin@shift-\the\c@pgf@counta)*\pgf@circ@res@step}{\pgf@circ@res@up},
                        rotate=\rot]{\mytext}
                \or % rotation -90
                    % left
                    \pgf@circ@count@c=\c@pgf@counta
                    \pgftext[top,
                        at=\pgfpoint{\pgf@circ@res@left}{\pgf@circ@res@up+(\pgf@circ@qfp@pin@shift-\the\c@pgf@counta)*\pgf@circ@res@step},
                        rotate=\rot]{\mytext}
                    % bottom
                    \pgf@circ@count@c=\numexpr\pgf@circ@count@b+\c@pgf@counta\relax
                    \pgftext[left,
                        at=\pgfpoint{\pgf@circ@res@left-(\pgf@circ@qfp@pin@shift-\the\c@pgf@counta)*\pgf@circ@res@step}{\pgf@circ@res@down},
                        rotate=\rot]{\mytext}
                    % right
                    \pgf@circ@count@c=\numexpr3*\pgf@circ@count@b-\c@pgf@counta+1\relax
                    \pgftext[bottom,
                        at=\pgfpoint{\pgf@circ@res@right}{\pgf@circ@res@up+(\pgf@circ@qfp@pin@shift-\the\c@pgf@counta)*\pgf@circ@res@step},
                        rotate=\rot]{\mytext}
                    % top
                    \pgf@circ@count@c=\numexpr3*\pgf@circ@count@b+\c@pgf@counta\relax
                    \pgftext[right,
                        at=\pgfpoint{\pgf@circ@res@right+(\pgf@circ@qfp@pin@shift-\the\c@pgf@counta)*\pgf@circ@res@step}{\pgf@circ@res@up},
                        rotate=\rot]{\mytext}
                \or %rotation 180
                    % left
                    \pgf@circ@count@c=\c@pgf@counta
                    \pgftext[right,
                        at=\pgfpoint{\pgf@circ@res@left}{\pgf@circ@res@up+(\pgf@circ@qfp@pin@shift-\the\c@pgf@counta)*\pgf@circ@res@step},
                        rotate=\rot]{\mytext}
                    % bottom
                    \pgf@circ@count@c=\numexpr\pgf@circ@count@b+\c@pgf@counta\relax
                    \pgftext[top,
                        at=\pgfpoint{\pgf@circ@res@left-(\pgf@circ@qfp@pin@shift-\the\c@pgf@counta)*\pgf@circ@res@step}{\pgf@circ@res@down},
                        rotate=\rot]{\mytext}
                    % right
                    \pgf@circ@count@c=\numexpr3*\pgf@circ@count@b-\c@pgf@counta+1\relax
                    \pgftext[left,
                        at=\pgfpoint{\pgf@circ@res@right}{\pgf@circ@res@up+(\pgf@circ@qfp@pin@shift-\the\c@pgf@counta)*\pgf@circ@res@step},
                        rotate=\rot]{\mytext}
                    % top
                    \pgf@circ@count@c=\numexpr3*\pgf@circ@count@b+\c@pgf@counta\relax
                    \pgftext[bottom,
                        at=\pgfpoint{\pgf@circ@res@right+(\pgf@circ@qfp@pin@shift-\the\c@pgf@counta)*\pgf@circ@res@step}{\pgf@circ@res@up},
                        rotate=\rot]{\mytext}
                \or % rotation +90
                    % left
                    \pgf@circ@count@c=\c@pgf@counta
                    \pgftext[bottom,
                        at=\pgfpoint{\pgf@circ@res@left}{\pgf@circ@res@up+(\pgf@circ@qfp@pin@shift-\the\c@pgf@counta)*\pgf@circ@res@step},
                        rotate=\rot]{\mytext}
                    % bottom
                    \pgf@circ@count@c=\numexpr\pgf@circ@count@b+\c@pgf@counta\relax
                    \pgftext[right,
                        at=\pgfpoint{\pgf@circ@res@left-(\pgf@circ@qfp@pin@shift-\the\c@pgf@counta)*\pgf@circ@res@step}{\pgf@circ@res@down},
                        rotate=\rot]{\mytext}
                    % right
                    \pgf@circ@count@c=\numexpr3*\pgf@circ@count@b-\c@pgf@counta+1\relax
                    \pgftext[top,
                        at=\pgfpoint{\pgf@circ@res@right}{\pgf@circ@res@up+(\pgf@circ@qfp@pin@shift-\the\c@pgf@counta)*\pgf@circ@res@step},
                        rotate=\rot]{\mytext}
                    % top
                    \pgf@circ@count@c=\numexpr3*\pgf@circ@count@b+\c@pgf@counta\relax
                    \pgftext[left,
                        at=\pgfpoint{\pgf@circ@res@right+(\pgf@circ@qfp@pin@shift-\the\c@pgf@counta)*\pgf@circ@res@step}{\pgf@circ@res@up},
                        rotate=\rot]{\mytext}
                \fi
                \advance\c@pgf@counta-1\relax%
                \repeatpgfmathloop
            \fi
            \endpgfscope
            \ifdim\pgf@circ@res@other>0pt
            \pgfscope
                \pgfsetlinewidth{\pgfkeysvalueof{/tikz/circuitikz/multipoles/external pins thickness}\pgflinewidth}
                \c@pgf@counta=\numpins%
                \divide\c@pgf@counta by 4 \pgf@circ@count@b=\c@pgf@counta
                \pgfmathloop%
                \ifnum\c@pgf@counta>0
                    \edef\padfrac{\pgfkeysvalueof{/tikz/circuitikz/multipoles/external pad fraction}}
                    \ifnum\padfrac>0
                        \pgf@circ@res@temp=\pgf@circ@res@step\divide\pgf@circ@res@temp by \padfrac
                        % left side pads
                        \pgfpathmoveto{\pgfpoint{\pgf@circ@res@left}{\pgf@circ@res@temp+\pgf@circ@res@up+(\pgf@circ@qfp@pin@shift-\the\c@pgf@counta)*\pgf@circ@res@step}}
                        \pgfpathlineto{\pgfpoint{\pgf@circ@res@left-\pgf@circ@res@other}{\pgf@circ@res@temp+\pgf@circ@res@up+(\pgf@circ@qfp@pin@shift-\the\c@pgf@counta)*\pgf@circ@res@step}}
                        \pgfpathlineto{\pgfpoint{\pgf@circ@res@left-\pgf@circ@res@other}{-\pgf@circ@res@temp+\pgf@circ@res@up+(\pgf@circ@qfp@pin@shift-\the\c@pgf@counta)*\pgf@circ@res@step}}
                        \pgfpathlineto{\pgfpoint{\pgf@circ@res@left}{-\pgf@circ@res@temp+\pgf@circ@res@up+(\pgf@circ@qfp@pin@shift-\the\c@pgf@counta)*\pgf@circ@res@step}}
                        % bottom side pads
                        \pgfpathmoveto{\pgfpoint{-\pgf@circ@res@temp+\pgf@circ@res@left-(\pgf@circ@qfp@pin@shift-\the\c@pgf@counta)*\pgf@circ@res@step}{\pgf@circ@res@down}}
                        \pgfpathlineto{\pgfpoint{-\pgf@circ@res@temp+\pgf@circ@res@left-(\pgf@circ@qfp@pin@shift-\the\c@pgf@counta)*\pgf@circ@res@step}{\pgf@circ@res@down-\pgf@circ@res@other}}
                        \pgfpathlineto{\pgfpoint{\pgf@circ@res@temp+\pgf@circ@res@left-(\pgf@circ@qfp@pin@shift-\the\c@pgf@counta)*\pgf@circ@res@step}{\pgf@circ@res@down-\pgf@circ@res@other}}
                        \pgfpathlineto{\pgfpoint{\pgf@circ@res@temp+\pgf@circ@res@left-(\pgf@circ@qfp@pin@shift-\the\c@pgf@counta)*\pgf@circ@res@step}{\pgf@circ@res@down}}
                        % right side pads
                        \pgfpathmoveto{\pgfpoint{\pgf@circ@res@right}{\pgf@circ@res@temp+\pgf@circ@res@up+(\pgf@circ@qfp@pin@shift-\the\c@pgf@counta)*\pgf@circ@res@step}}
                        \pgfpathlineto{\pgfpoint{\pgf@circ@res@right+\pgf@circ@res@other}{\pgf@circ@res@temp+\pgf@circ@res@up+(\pgf@circ@qfp@pin@shift-\the\c@pgf@counta)*\pgf@circ@res@step}}
                        \pgfpathlineto{\pgfpoint{\pgf@circ@res@right+\pgf@circ@res@other}{-\pgf@circ@res@temp+\pgf@circ@res@up+(\pgf@circ@qfp@pin@shift-\the\c@pgf@counta)*\pgf@circ@res@step}}
                        \pgfpathlineto{\pgfpoint{\pgf@circ@res@right}{-\pgf@circ@res@temp+\pgf@circ@res@up+(\pgf@circ@qfp@pin@shift-\the\c@pgf@counta)*\pgf@circ@res@step}}
                        % top side pads
                        \pgfpathmoveto{\pgfpoint{\pgf@circ@res@temp+\pgf@circ@res@right+(\pgf@circ@qfp@pin@shift-\the\c@pgf@counta)*\pgf@circ@res@step}{\pgf@circ@res@up}}
                        \pgfpathlineto{\pgfpoint{\pgf@circ@res@temp+\pgf@circ@res@right+(\pgf@circ@qfp@pin@shift-\the\c@pgf@counta)*\pgf@circ@res@step}{\pgf@circ@res@up+\pgf@circ@res@other}}
                        \pgfpathlineto{\pgfpoint{-\pgf@circ@res@temp+\pgf@circ@res@right+(\pgf@circ@qfp@pin@shift-\the\c@pgf@counta)*\pgf@circ@res@step}{\pgf@circ@res@up+\pgf@circ@res@other}}
                        \pgfpathlineto{\pgfpoint{-\pgf@circ@res@temp+\pgf@circ@res@right+(\pgf@circ@qfp@pin@shift-\the\c@pgf@counta)*\pgf@circ@res@step}{\pgf@circ@res@up}}
                    \else
                        % left side pins
                        \pgfpathmoveto{\pgfpoint{\pgf@circ@res@left}{\pgf@circ@res@up+(\pgf@circ@qfp@pin@shift-\the\c@pgf@counta)*\pgf@circ@res@step}}
                        \pgfpathlineto{\pgfpoint{\pgf@circ@res@left-\pgf@circ@res@other}{\pgf@circ@res@up+(\pgf@circ@qfp@pin@shift-\the\c@pgf@counta)*\pgf@circ@res@step}}
                        % bottom side pins
                        \pgfpathmoveto{\pgfpoint{\pgf@circ@res@left-(\pgf@circ@qfp@pin@shift-\the\c@pgf@counta)*\pgf@circ@res@step}{\pgf@circ@res@down}}
                        \pgfpathlineto{\pgfpoint{\pgf@circ@res@left-(\pgf@circ@qfp@pin@shift-\the\c@pgf@counta)*\pgf@circ@res@step}{\pgf@circ@res@down-\pgf@circ@res@other}}
                        % right side pins
                        \pgfpathmoveto{\pgfpoint{\pgf@circ@res@right}{\pgf@circ@res@up+(\pgf@circ@qfp@pin@shift-\the\c@pgf@counta)*\pgf@circ@res@step}}
                        \pgfpathlineto{\pgfpoint{\pgf@circ@res@right+\pgf@circ@res@other}{\pgf@circ@res@up+(\pgf@circ@qfp@pin@shift-\the\c@pgf@counta)*\pgf@circ@res@step}}
                        % top side pins
                        \pgfpathmoveto{\pgfpoint{\pgf@circ@res@right+(\pgf@circ@qfp@pin@shift-\the\c@pgf@counta)*\pgf@circ@res@step}{\pgf@circ@res@up}}
                        \pgfpathlineto{\pgfpoint{\pgf@circ@res@right+(\pgf@circ@qfp@pin@shift-\the\c@pgf@counta)*\pgf@circ@res@step}{\pgf@circ@res@up+\pgf@circ@res@other}}
                    \fi
                    \advance\c@pgf@counta-1\relax%
                \repeatpgfmathloop
                \pgfusepath{stroke}
            \endpgfscope
            \fi
        }%
        % \pgf@sh@s@<name of the shape here> contains all the code for the shape
        % and is executed just before a node is drawn.
        \pgfutil@g@addto@macro\pgf@sh@s@qfpchip{%
            % Start with the maximum pin number and go backwards.
            \c@pgf@counta=\numpins%
            \pgfmathloop%
            \ifnum\c@pgf@counta>0
                \expandafter\xdef\csname pgf@anchor@qfpchip@pin\space\the\c@pgf@counta\endcsname{%
                    \noexpand\pgf@circ@qfppinanchor{\the\c@pgf@counta}{1}%
                }
                \expandafter\xdef\csname pgf@anchor@qfpchip@bpin\space\the\c@pgf@counta\endcsname{%
                    \noexpand\pgf@circ@qfppinanchor{\the\c@pgf@counta}{0}%
                }
                \advance\c@pgf@counta-1\relax%
                \repeatpgfmathloop%
            }%
        }

%% anchors for DIP
\def\pgf@circ@dippinanchor#1#2{% #1: pin number #2: 0 for border pin, 1 for external pin
    \c@pgf@countc=\numpins\relax
    \divide\c@pgf@countc by 2
    \ifnum #1 > \the\c@pgf@countc
        % right side
        \pgfpoint{\width/2+#2*\extshift}{-\height/2+(\pgf@circ@dip@pin@shift-\c@pgf@countc+#1-1)*\chipspacing}
    \else
        \pgfpoint{-\width/2-#2*\extshift}{\height/2+(\pgf@circ@dip@pin@shift-#1)*\chipspacing}
\fi
}

%% anchors for QFP
\def\pgf@circ@qfppinanchor#1#2{% #1: pin number #2: 0 for border pin, 1 for external pin
    \c@pgf@countc=\numpins\relax
    \divide\c@pgf@countc by 4
    \ifnum #1 > \the\c@pgf@countc
        \c@pgf@countb=\c@pgf@countc \multiply \c@pgf@countb by 2
        \ifnum #1 > \the\c@pgf@countb
            \c@pgf@countb=\c@pgf@countc \multiply \c@pgf@countb by 3
            \ifnum #1 > \the\c@pgf@countb
                % 3*npins/4 < pin, top side
                \pgfpoint{\width/2+(\pgf@circ@qfp@pin@shift+\c@pgf@countb-#1)*\chipspacing}{\height/2+#2*\extshift}%
            \else
                % 2*npins/4 < pin <= 3*npins/4, right side
                \pgfpoint{\width/2+#2*\extshift}{\height/2+(\pgf@circ@qfp@pin@shift-\c@pgf@countb+#1-1)*\chipspacing}%
            \fi
        \else
            %  npins/4 < pin <= 2*npins/4, bottom side
            \pgfpoint{\width/2+(\pgf@circ@qfp@pin@shift-\c@pgf@countb+#1-1)*\chipspacing}{-\height/2-#2*\extshift}%
        \fi
    \else
        % <= npins/4, left side
        \pgfpoint{-\width/2-#2*\extshift}{\height/2+(\pgf@circ@qfp@pin@shift-#1)*\chipspacing}%
    \fi
}


%% Rotary Switch

\pgfdeclareshape{rotaryswitch}
{
    \savedanchor\northeast{%
        % this strange value makes the 2-pole rotary switch equal to the 2 poles cute spdt
        % the magic number is 0.25/cos(35)
        % try to recalculate it for the actual switch
        \pgf@circ@res@temp=\ctikzvalof{tripoles/spdt/width}\pgf@circ@Rlen
        \pgf@circ@res@temp=.3052\pgf@circ@res@temp
        \edef\a{\ctikzvalof{multipoles/rotary/angle}}
        \edef\r{\ctikzvalof{nodes width}}
        \pgfmathsetlength{\pgf@y}{\r*\pgf@circ@Rlen +(\a>90 ? 2 : 2*sin(\a))*\pgf@circ@res@temp}
        \pgfmathsetlength{\pgf@x}{\r*\pgf@circ@Rlen + \pgf@circ@res@temp}
    }
    \savedanchor\northwest{%
        % this strange value makes the 2-pole rotary switch equal to the 2 poles cute spdt
        % the magic number is 0.25/cos(35)
        % try to recalculate it for the actual switch
        \pgf@circ@res@temp=\ctikzvalof{tripoles/spdt/width}\pgf@circ@Rlen
        \pgf@circ@res@temp=.3052\pgf@circ@res@temp
        \edef\a{\ctikzvalof{multipoles/rotary/angle}}
        \edef\r{\ctikzvalof{nodes width}}
        \pgfmathsetlength{\pgf@y}{\r*\pgf@circ@Rlen +(\a>90 ? 2 : 2*sin(\a))*\pgf@circ@res@temp}
        \pgfmathsetlength{\pgf@x}{-\r*\pgf@circ@Rlen - (\a<90 ? 1 : 1-2*cos(\a))*\pgf@circ@res@temp}
    }
    \savedanchor\central{%
        % this strange value makes the 2-pole rotary switch equal to the 2 poles cute spdt
        % the magic number is 0.25/cos(35)
        % try to recalculate it for the actual switch
        \pgf@circ@res@temp=\ctikzvalof{tripoles/spdt/width}\pgf@circ@Rlen
        \pgf@circ@res@temp=.3052\pgf@circ@res@temp
        \edef\a{\ctikzvalof{multipoles/rotary/angle}}
        \edef\r{\ctikzvalof{nodes width}}
        \pgfmathsetlength{\pgf@y}{\r*\pgf@circ@Rlen +(\a>90 ? 2 : 2*sin(\a))*\pgf@circ@res@temp}
        \pgfmathsetlength{\pgf@x}{(\a<90 ? 0 : cos(\a))*\pgf@circ@res@temp}
    }
    % external square limits
    \savedanchor\extnorthwest{%
        \pgf@x=-\ctikzvalof{tripoles/spdt/width}\pgf@circ@Rlen
        % this strange value makes the 2-pole rotary switch equal to the 2 poles cute spdt
        \pgf@x=.3052\pgf@x % the magic number is 0.25/cos(35)
        \pgf@x=2.5\pgf@x % external square size
        \pgf@y=-\pgf@x %square thing when angle=180?
    }
    \saveddimen{\width}{\pgfmathsetlength\pgf@x{0.3052*\pgf@circ@Rlen*\ctikzvalof{tripoles/spdt/width}}}
    % radius of the connector
    % This is the radius of the "ocirc" shape (see pgfcircshapes.tex)
    \saveddimen{\radius}{\pgfmathsetlength\pgf@x{\pgf@circ@Rlen*\ctikzvalof{nodes width}}}
    % shapename
    \savedmacro{\thisshape}{\def\thisshape{\tikz@fig@name}}
    % shape type
    \savedmacro{\cshape}{\def\cshape{\ctikzvalof{multipoles/rotary/shape}}}
    \savedmacro{\channels}{\def\channels{\ctikzvalof{multipoles/rotary/channels}}}
    \savedmacro{\angle}{\def\angle{\ctikzvalof{multipoles/rotary/angle}}}
    \savedmacro{\wiper}{\def\wiper{\ctikzvalof{multipoles/rotary/wiper}}}
    \savedmacro{\stepa}{\pgfmathsetmacro{\stepa}{2*\ctikzvalof{multipoles/rotary/angle}/(\ctikzvalof{multipoles/rotary/channels}-1)}}
    % mid of the lever, to stack switches
    %\anchor{mid}{\midlever}
    \anchor{mid}{\northwest
        \pgf@circ@res@temp=-\pgf@x
        \pgfmathsetlength{\pgf@x}{\pgf@circ@res@temp*(-1+cos(\wiper))}
        \pgfmathsetlength{\pgf@y}{\pgf@circ@res@temp*sin(\wiper)}
    }
    % center anchors
    \anchor{cin}{ \northwest \pgf@y=0pt\advance\pgf@x by \radius}
    % horizontal angles
    \anchor{in}{ \northwest \pgf@y=0pt}
    \anchor{ain}{ \northwest \pgf@y=0pt}

    \anchor{center}{ \central \pgf@y=0pt }
    \anchor{east}{ \northeast \pgf@y=0pt }
    \anchor{west}{ \northwest \pgf@y=0pt }
    \anchor{south}{ \central \pgf@y=-\pgf@y }
    \anchor{north}{ \central }
    \anchor{south west}{ \northwest \pgf@y=-\pgf@y }
    \anchor{north east}{ \northeast }
    \anchor{north west}{ \northwest }
    \anchor{south east}{ \northeast \pgf@y=-\pgf@y }

    \anchor{ext center}{ \pgf@y=0pt \pgf@x=0pt \advance\pgf@x by -\width}
    \anchor{ext east}{ \extnorthwest \pgf@y=0pt \pgf@x=-\pgf@x \advance\pgf@x by -\width}
    \anchor{ext west}{ \extnorthwest \pgf@y=0pt \advance\pgf@x by -\width}
    \anchor{ext south}{ \extnorthwest \pgf@x=0pt \pgf@y=-\pgf@y \advance\pgf@x by -\width}
    \anchor{ext north}{ \extnorthwest \pgf@x=0pt \advance\pgf@x by -\width}
    \anchor{ext south west}{ \extnorthwest \pgf@y=-\pgf@y \advance\pgf@x by -\width}
    \anchor{ext north east}{ \extnorthwest \pgf@x=-\pgf@x \advance\pgf@x by -\width}
    \anchor{ext north west}{ \extnorthwest \advance\pgf@x by -\width}
    \anchor{ext south east}{ \extnorthwest \pgf@x=-\pgf@x \pgf@y=-\pgf@y \advance\pgf@x by -\width}

    \backgroundpath{
        \pgfsetcolor{\pgfkeysvalueof{/tikz/circuitikz/color}}
        \pgf@circ@res@right = \width
        \pgf@circ@res@left = -\width

        \pgfscope %wiper
        % This is the radius of the "ocirc" shape (see pgfcircshapes.tex)
        \pgf@circ@res@temp=\radius\relax
        \pgf@circ@res@temp=\ctikzvalof{multipoles/rotary/thickness}\pgf@circ@res@temp
        \pgfsetlinewidth{2\pgf@circ@res@temp}
        \pgfpathmoveto{\pgfpoint{\pgf@circ@res@left}{0pt}}
        \pgfpathlineto{\pgfpointadd{\pgfpoint{\pgf@circ@res@left}{0pt}}{\pgfpointpolar{\wiper}{2\pgf@circ@res@right}}}
        \pgfsetroundcap\pgfusepath{draw}
        \endpgfscope

        \ifpgf@circ@rotaryarrow
            \pgfscope % arrow
                \ifpgf@circ@rotaryarrow@ccw\pgfsetarrowsstart{latexslim}\fi
                \pgfsetlinewidth{\pgfkeysvalueof{/tikz/circuitikz/bipoles/thickness}\pgflinewidth}
                \pgftransformshift{\pgfpoint{\pgf@circ@res@left}{0pt}} % center of cin node
                \pgftransformrotate{\wiper}
                \pgfpathmoveto{\pgfpointpolar{50}{1.0\pgf@circ@res@right}}
                \pgfpatharc{50}{-50}{1.0\pgf@circ@res@right}
                \ifpgf@circ@rotaryarrow@cw\pgfsetarrowsend{latexslim}\fi
                \pgfusepath{draw}
            \endpgfscope
        \fi

        % \typeout{CHANNELS\space\channels\space ANGLE\space\angle}
        \c@pgf@counta=\channels\relax
        \pgfmathsetmacro{\currenta}{-\angle}
        \pgfmathloop%
        \ifnum\c@pgf@counta>0
            % \typeout{LOOP\space\the\c@pgf@counta\space CURRENTA\space\currenta\space RIGHT\space\the\pgf@circ@res@right}
            \pgfscope
                \pgftransformshift{\pgfpointadd{\pgfpoint{\pgf@circ@res@left}{0pt}}{\pgfpointpolar{\currenta}{2\pgf@circ@res@right}}}
                \pgfnode{\cshape}{center}{}{\thisshape-out \the\c@pgf@counta}{\pgfusepath{stroke}}
            \endpgfscope
            \pgfmathsetmacro{\currenta}{\currenta+\stepa}
            \advance\c@pgf@counta by -1\relax%
        \repeatpgfmathloop

        \pgfscope % input
        \pgftransformshift{\pgfpoint{\pgf@circ@res@left}{0pt}}
        \pgfnode{\cshape}{center}{}{\thisshape-in}{\pgfusepath{stroke}}
        \endpgfscope
    }
    % \pgf@sh@s@<name of the shape here> contains all the code for the shape
    % and is executed just before a node is drawn.
    \pgfutil@g@addto@macro\pgf@sh@s@rotaryswitch{%
        % Start with the maximum pin number and go backwards.
        \c@pgf@counta=\channels\relax
        \pgfmathloop%
        \ifnum\c@pgf@counta>0
        % we will create two anchors per pin: the "normal one" like `pin 1` for the
        % electrical contact, and the "border one" like `bpin 1` for labels.
        % they will coincide if `external pins width` is set to 0.
        \expandafter\xdef\csname pgf@anchor@rotaryswitch@out\space\the\c@pgf@counta\endcsname{%
            \noexpand\pgf@circ@rotaryanchor{\the\c@pgf@counta}{1}{0}%
        }
        \expandafter\xdef\csname pgf@anchor@rotaryswitch@cout\space\the\c@pgf@counta\endcsname{%
            \noexpand\pgf@circ@rotaryanchor{\the\c@pgf@counta}{0}{0}%
        }
        \expandafter\xdef\csname pgf@anchor@rotaryswitch@aout\space\the\c@pgf@counta\endcsname{%
            \noexpand\pgf@circ@rotaryanchor{\the\c@pgf@counta}{0}{1}%
        }
        \expandafter\xdef\csname pgf@anchor@rotaryswitch@sqout\space\the\c@pgf@counta\endcsname{%
            \noexpand\pgf@circ@rotarysqanchor{\the\c@pgf@counta}{0}%
        }
        \advance\c@pgf@counta by -1\relax%
        \repeatpgfmathloop%
    }%
}

\def\pgf@circ@rotaryanchor#1#2#3{% #1: numero del pin; #2: 1 - x pos, 0 - center; #3 0: inner, 1 outer
    \pgf@circ@res@temp=\width
    \pgfmathsetmacro{\myangle}{\angle-(#1-1)*\stepa}
    \pgfmathsetlength{\pgf@x}{2*(\pgf@circ@res@temp+#3*\radius/2)*cos(\myangle))+#2*\radius}
    \pgfmathsetlength{\pgf@y}{2*(\pgf@circ@res@temp+#3*\radius/2)*sin(\myangle)}
    \advance\pgf@x by -\pgf@circ@res@temp
}

\def\pgf@circ@rotarysqanchor#1{% external square anchors
    \pgf@circ@res@temp=\width
    \pgfmathsetmacro{\myangle}{\angle-(#1-1)*\stepa}
    \pgfpointborderrectangle{\pgfpointpolar{\myangle}{1pt}}{\pgfpoint{2.5\pgf@circ@res@temp}{2.5\pgf@circ@res@temp}}
    \advance\pgf@x by -\pgf@circ@res@temp
}
