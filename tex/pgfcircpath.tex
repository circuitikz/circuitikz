% Copyright 2018-2020 by Romano Giannetti
% Copyright 2015-2020 by Stefan Lindner
% Copyright 2013-2020 by Stefan Erhardt
% Copyright 2007-2020 by Massimo Redaelli
%
% This file may be distributed and/or modified
%
% 1. under the LaTeX Project Public License and/or
% 2. under the GNU Public License.
%
% See the files gpl-3.0_license.txt and lppl-1-3c_license.txt for more details.


\def\pgf@circ@direction{0.0}

% swap two coordinates
\def\pgfcirc@swap@coordinates#1#2{%
    coordinate (pgfcirc@tmp@swap) at (#1)
    coordinate (#1) at (#2)
    coordinate (#2) at (pgfcirc@tmp@swap)
}

% Names
\ctikzset{name/.style = { n=#1 } } %%%%%%%%%%%%%%%%%%%%%%%%%%%%%%%%%%%@@@
\ctikzset{n/.code = {
	\pgfkeys{/tikz/circuitikz/bipole/name=#1}
}}

% Reflect the node along
\ctikzset{mirrored/.is choice}
\ctikzset{mirror value/.initial=1}
\ctikzset{mirrored/true/.code = {\ctikzsetvalof{mirror value}{-1}} }
\ctikzset{mirrored/false/.code = {\ctikzsetvalof{mirror value}{1}} }
\ctikzset{mirror/.style = {/tikz/circuitikz/mirrored=true}}

% Invert node along path
\ctikzset{inverted/.is choice}
\ctikzset{invert value/.initial=1}
\ctikzset{inverted/true/.code = {\ctikzsetvalof{invert value}{-1}\pgf@circuit@bipole@invertedtrue}}
\ctikzset{inverted/false/.code = {\ctikzsetvalof{invert value}{1}\pgf@circuit@bipole@invertedfalse}}
\ctikzset{invert/.style = {/tikz/circuitikz/inverted=true}}
\newif\ifpgf@circuit@bipole@inverted
\ctikzset{bipole/inverted/.is if=pgf@circuit@bipole@inverted}

\newif\ifpgf@circuit@bipole@voltage@backward
\ctikzset{bipole/voltage/direction/.is choice}
\ctikzset{bipole/voltage/direction/forward/.code={\pgf@circuit@bipole@voltage@backwardfalse}}
\ctikzset{bipole/voltage/direction/backward/.code={\pgf@circuit@bipole@voltage@backwardtrue}}

% Initialize paths
\def\pgfcircresetpath{
    \ctikzset{bipole/name=, bipole/label/name=, bipole/label/position=90, ,bipole/annotation/name=, bipole/annotation/position=-90,
        bipole/inverted=false, bipole/kind=,
        bipole/voltage/direction=backward, bipole/voltage/label/name=, bipole/voltage/position=below,
        bipole/nodes/left=none, bipole/nodes/right=none, bipole/is voltage=false,bipole/is voltageoutsideofsymbol=false,bipole/is strokedsymbol=false,
        bipole/is current=false, bipole/current/label/name=, bipole/current/x position=after,
        bipole/current/y position=above, bipole/current/direction=forward,
        mirrored=false
    }
}

%
% expandable IF for the extra nodes (thanks to Henri Menke)
% see https://chat.stackexchange.com/transcript/message/56560808#56560808
%
\def\pgfcirc@if@has@i{%
    \ifpgfcirc@has@i
        \expandafter\pgfutil@firstoftwo
    \else
        \expandafter\pgfutil@secondoftwo
    \fi}
\def\pgfcirc@if@has@v{%
    \ifpgfcirc@has@v
        \expandafter\pgfutil@firstoftwo
    \else
        \expandafter\pgfutil@secondoftwo
    \fi}
\def\pgfcirc@if@has@f{%
    \ifpgfcirc@has@f
        \expandafter\pgfutil@firstoftwo
    \else
        \expandafter\pgfutil@secondoftwo
    \fi}



%% Generic bipole path
\def\pgf@circ@bipole@path#1#2{
    % Create a bipole path from the shapes defined with \pgfcircdeclarebipole
    % or \pgfcircdeclarebipolescaled; the node shapes are named with a "shape"
    % appended to the main (path-style) name
    % #1 path-style node name
    % #2 the argument passed from the to-path structure; don't touch
    %
    % Example:
    % \def\pgf@circ@capacitor@path#1{\pgf@circ@bipole@path{capacitor}{#1}}
    %
    \pgf@circ@bipole@path@base{shape}{}{#1}{#2}
}
%
% this is used for components that are mainly node-style but have a path-style form
%
\def\pgfcirc@node@to@path#1#2#3{%
    % add a path-style component based on a node-style one without mangling the name
    % of the shape.
    % #1: node-type shape name (existing)
    % #2: path-type name (to be created)
    % #3: additional options to add to the path style
    %
    \expandafter\def\csname pgf@circ@#1@path\endcsname##1{\pgf@circ@bipole@path@base{}{##1}{#1}{}}%
    \compattikzset{#2/.style = {\circuitikzbasekey,
        /tikz/to path=\csname pgf@circ@#1@path\endcsname{##1},
        #3}}%
    \ctikzset{bipoles/#1/height/.initial=1}%
}
%%
%% ultra-generic bipole path
%% I am not sure what the last argument is needed for, but don't touch it or everything explodes
%%
\def\pgf@circ@bipole@path@base#1#2#3#4{%
    %
    % Create a path-style component based on a node-style shape
    % #1: postfix to be added to the name path to obtain the main shape name
    % #2: text to be passed as text to the node
    % #3: name of the bipole component
    % #4: this will be filled by the argument of the to-path
    %
    \pgfextra{
        \ctikzset{bipole/kind = #3}
        \edef\pgf@temp{\ctikzvalof{bipole/name}}
        \def\pgf@circ@temp{}
        \ifx\pgf@temp\pgf@circ@temp % if it has not a name
            \pgfmathrandominteger{\pgf@circ@rand}{1000}{9999}
            \ctikzset{bipole/name = pgfcirc@#3\pgf@circ@rand} % create it (re-usage should not create problem, but...)
            \edef\pgfcirc@a@prefix{pgfcirc}% do not pollute the namespace for nothing
        \else
            \edef\pgfcirc@a@prefix{\ctikzvalof{bipole/name}}% for exporting v-i-f anchors
        \fi
    }
    % save start and stop values
    % notice that we DO NOT MOVE the path position at all!
    coordinate (\ctikzvalof{bipole/name}start) at (\tikztostart)
    coordinate (\ctikzvalof{bipole/name}end) at (\tikztotarget)
    \pgfextra{
        % find the direction (angle) of the path
        \pgfmathanglebetweenpoints{\pgfpointanchor{\ctikzvalof{bipole/name}start}{center}}
            {\pgfpointanchor{\ctikzvalof{bipole/name}end}{center}}
        \edef\pgf@circ@direction{\pgfmathresult}
        \expandafter\xdef\csname pgfcirc@\pgfcirc@a@prefix-direction\endcsname{\pgf@circ@direction}
    }
    % position the component in the middle of the path. We DO NOT MOVE the current position!
    node[#3#1, rotate=\pgf@circ@direction, yscale=\ctikzvalof{mirror value},
        xscale=\ctikzvalof{invert value}] (\ctikzvalof{bipole/name})
        at ($(\tikztostart) ! .5 ! (\tikztotarget)$) {#2}
    % set start and end labels
    \ifpgf@circuit@bipole@inverted
        \ifcsname pgf@anchor@#3#1@pathstart\endcsname%if special path-anchors are defined, use them!
            coordinate	(pgfcirc@anchorstartnode) at (\ctikzvalof{bipole/name}.pathend)
            coordinate	(pgfcirc@anchorendnode) at (\ctikzvalof{bipole/name}.pathstart)
        \else
            coordinate	(pgfcirc@anchorstartnode) at (\ctikzvalof{bipole/name}.right)
            coordinate	(pgfcirc@anchorendnode) at (\ctikzvalof{bipole/name}.left)
        \fi
        \else
        \ifcsname pgf@anchor@#3#1@pathstart\endcsname%if special path-anchors are defined, use them!
            coordinate	(pgfcirc@anchorstartnode) at (\ctikzvalof{bipole/name}.pathstart)
            coordinate	(pgfcirc@anchorendnode) at (\ctikzvalof{bipole/name}.pathend)
        \else
            coordinate	(pgfcirc@anchorstartnode) at (\ctikzvalof{bipole/name}.left)
            coordinate	(pgfcirc@anchorendnode) at (\ctikzvalof{bipole/name}.right)
        \fi
    \fi
    % draw the leads unless it's an open circuit
    % stop at the component
    \pgfextra{\def\pgf@temp{open}\def\pgf@circ@temp{#3}}
    \ifx\pgf@temp\pgf@circ@temp  % if it is an open do nothing
    \else
        % it is important to start the path with -- to have correct line joins!
        -- (\tikztostart) -- (pgfcirc@anchorstartnode)
    \fi
    % Add all the "ornaments": labels, annotations, voltages, currents and flows
    \drawpoles
    \pgf@circ@ifkeyempty{bipole/label/name}\else\pgf@circ@drawlabels{label}\fi
    \pgf@circ@ifkeyempty{bipole/annotation/name}\else\pgf@circ@drawlabels{annotation}\fi
    % the following  must be made in their own path scope to avoid crash in TikZ 3.1.8/3.1.8a
    % it should be logically safe for older version too --- even if TikZ reverted the change
    % use explandable ifs too, thanks to Henri Menke
    {\pgfcirc@if@has@v{\pgf@circ@drawvoltage}{}}%
    {\pgfcirc@if@has@i{\pgf@circ@drawcurrent}{}}%
    {\pgfcirc@if@has@f{\pgf@circ@drawflow}{}}%
    % finish the path from the component to the final target
    % you never know --- re-set \pgf@temp to detect open
    \pgfextra{\def\pgf@temp{open}\def\pgf@circ@temp{#3}}
    \ifx\pgf@temp\pgf@circ@temp  % if it is an open do nothing
        (\tikztotarget)
    \else
        (pgfcirc@anchorendnode)  -- (\tikztotarget)
    \fi
    % reset internal circuit keys
    \pgfextra{\pgfcircresetpath}
    %draw pending nodes an path
    \tikztonodes
}

%%%%%%%%%%%%%%%%%%%%%%%%%%%%%
%% Handling of terminals

\ctikzset{bipole/nodes/.is family}
\ctikzset{bipole/nodes/left/.initial=none}
\ctikzset{bipole/nodes/right/.initial=none}
\tikzset{bipole nodes/.style n args={2}{%
    \circuitikzbasekey/bipole/nodes/left=#1,
    \circuitikzbasekey/bipole/nodes/right=#2
    }
}

%% Easily usable styles

\ctikzset{o-o/.style = {\circuitikzbasekey/bipole/nodes/left=ocirc, \circuitikzbasekey/bipole/nodes/right=ocirc}}
\ctikzset{-o/.style = {\circuitikzbasekey/bipole/nodes/left=none, \circuitikzbasekey/bipole/nodes/right=ocirc}}
\ctikzset{o-/.style = {\circuitikzbasekey/bipole/nodes/left=ocirc, \circuitikzbasekey/bipole/nodes/right=none}}
\ctikzset{*-o/.style = {\circuitikzbasekey/bipole/nodes/left=circ, \circuitikzbasekey/bipole/nodes/right=ocirc}}
\ctikzset{o-*/.style = {\circuitikzbasekey/bipole/nodes/left=ocirc, \circuitikzbasekey/bipole/nodes/right=circ}}
\ctikzset{d-o/.style = {\circuitikzbasekey/bipole/nodes/left=diamondpole, \circuitikzbasekey/bipole/nodes/right=ocirc}}
\ctikzset{o-d/.style = {\circuitikzbasekey/bipole/nodes/left=ocirc, \circuitikzbasekey/bipole/nodes/right=diamondpole}}
\ctikzset{*-/.style = {\circuitikzbasekey/bipole/nodes/left=circ, \circuitikzbasekey/bipole/nodes/right=none}}
\ctikzset{-*/.style = {\circuitikzbasekey/bipole/nodes/left=none, \circuitikzbasekey/bipole/nodes/right=circ}}
\ctikzset{d-/.style = {\circuitikzbasekey/bipole/nodes/left=diamondpole, \circuitikzbasekey/bipole/nodes/right=none}}
\ctikzset{-d/.style = {\circuitikzbasekey/bipole/nodes/left=none, \circuitikzbasekey/bipole/nodes/right=diamondpole}}
\ctikzset{*-*/.style = {\circuitikzbasekey/bipole/nodes/left=circ, \circuitikzbasekey/bipole/nodes/right=circ}}
\ctikzset{d-*/.style = {\circuitikzbasekey/bipole/nodes/left=diamondpole, \circuitikzbasekey/bipole/nodes/right=circ}}
\ctikzset{*-d/.style = {\circuitikzbasekey/bipole/nodes/left=circ, \circuitikzbasekey/bipole/nodes/right=diamondpole}}
\ctikzset{d-d/.style = {\circuitikzbasekey/bipole/nodes/left=diamondpole, \circuitikzbasekey/bipole/nodes/right=diamondpole}}

% rectjoinfill workarounds

\ctikzset{.-/.style = {\circuitikzbasekey/bipole/nodes/left=rectjoinfill, \circuitikzbasekey/bipole/nodes/right=none}}
\ctikzset{.-*/.style = {\circuitikzbasekey/bipole/nodes/left=rectjoinfill, \circuitikzbasekey/bipole/nodes/right=circ}}
\ctikzset{.-o/.style = {\circuitikzbasekey/bipole/nodes/left=rectjoinfill, \circuitikzbasekey/bipole/nodes/right=ocirc}}
\ctikzset{.-d/.style = {\circuitikzbasekey/bipole/nodes/left=rectjoinfill, \circuitikzbasekey/bipole/nodes/right=diamondpole}}
\ctikzset{-./.style = {\circuitikzbasekey/bipole/nodes/left=none, \circuitikzbasekey/bipole/nodes/right=rectjoinfill}}
\ctikzset{o-./.style = {\circuitikzbasekey/bipole/nodes/left=ocirc, \circuitikzbasekey/bipole/nodes/right=rectjoinfill}}
\ctikzset{*-./.style = {\circuitikzbasekey/bipole/nodes/left=circ, \circuitikzbasekey/bipole/nodes/right=rectjoinfill}}
\ctikzset{d-./.style = {\circuitikzbasekey/bipole/nodes/left=diamondpole, \circuitikzbasekey/bipole/nodes/right=rectjoinfill}}
\ctikzset{.-./.style = {\circuitikzbasekey/bipole/nodes/left=rectjoinfill, \circuitikzbasekey/bipole/nodes/right=rectjoinfill}}

\tikzset{reversed/.style = {\circuitikzbasekey/bipole/inverted=true}}

\def\drawpoles{
    \pgfextra{ \edef\pgf@circ@temp{\ctikzvalof{bipole/nodes/left}} \def\pgf@temp{none}}
    \ifx\pgf@temp\pgf@circ@temp\else(\tikztostart) node[\pgf@circ@temp] {}\fi
    \pgfextra{ \edef\pgf@circ@temp{\ctikzvalof{bipole/nodes/right}} }
    \ifx\pgf@temp\pgf@circ@temp\else(\tikztotarget) node[\pgf@circ@temp] {}\fi
}

%% Macros for path and style activation for bipoles or path-style

\def\comnpatname{\ifpgf@circuit@compat *\else\fi}
\def\compattikzset#1{%
    % \typeout{BIPOLEDEF:\space \detokenize{#1}}%
    \tikzset{\comnpatname#1}}
%
% this one is for normal definition: path to style, directly
% the first parameter (#1) here is l,v,i (l=..., v=..., i=...)
% the last parameter are options to be inserted in the "to path" definition
%
\def\pgfcirc@path@to@style#1#2#3#4{% using #1 as label, assign \pgf@circ@#2@path to style #3
    \compattikzset{#3/.style={\circuitikzbasekey, #4, /tikz/to path=\csname pgf@circ@#2@path\endcsname, #1=##1}}%
}
% this one create a alias style from a node definition
\def\pgfcirc@node@to@style#1#2#3#4{% using #1 as label, assign \pgf@circ@bipole@path{#2} to style #3
    \compattikzset{#3/.style={\circuitikzbasekey, #4, /tikz/to path=\pgf@circ@bipole@path{#2}, #1=##1}}%
}
% this create an alias style
\def\pgfcirc@style@to@style#1#2{% alias style #1 to style #2
    \compattikzset{#2/.style={\comnpatname #1 = ##1}}%
}
% this create an alias style, changing the labelling
\def\pgfcirc@style@to@style@label#1#2#3{% alias style #1 to style #2
    \compattikzset{#2/.style={\comnpatname #1, #3 = ##1}}%
}
% create a bipole
\def\pgfcirc@activate@bipole#1#2#3#4{% path name, base node name, style name
    \expandafter\def\csname pgf@circ@#2@path\endcsname##1{\pgf@circ@bipole@path{#3}{##1}}%
    \pgfcirc@path@to@style{#1}{#2}{#4}{}% no options here, let's see
}
\def\pgfcirc@activate@bipole@simple#1#2{\pgfcirc@activate@bipole{#1}{#2}{#2}{#2}}
% create a bipole with options
\def\pgfcirc@activate@bipole@opt#1#2#3#4#5{% path name, base node name, style name
    \expandafter\def\csname pgf@circ@#2@path\endcsname##1{\pgf@circ@bipole@path{#3}{##1}}%
    \pgfcirc@path@to@style{#1}{#2}{#4}{#5}% no options here, let's see
}
\def\pgfcirc@activate@bipole@simple@opt#1#2#3{\pgfcirc@activate@bipole@opt{#1}{#2}{#2}{#2}{#3}}


%% New system, for simple object
%% \pgfcirc@activate@bipole@simple{l}{mass}
%% New system, different names
%% The old system is the following
%% 1 - define just the pgf@circ@path@whatever#1
%% (see for example the variable one)
%% 2 - set the style
%% \compattikzset{resistor/.style= {\circuitikzbasekey, /tikz/to path=\pgf@circ@resistor@path, l=#1}}

%% Path definition with the new mechanism


%% GENERICS
\def\pgf@circ@empty@path#1{}
\pgfcirc@activate@bipole@simple{l}{generic}
\pgfcirc@activate@bipole@simple{l}{ageneric}
\pgfcirc@activate@bipole@simple{l}{tgeneric}
\pgfcirc@activate@bipole@simple{l}{xgeneric}
\pgfcirc@activate@bipole@simple{l}{fullgeneric}
\pgfcirc@activate@bipole@simple{l}{tfullgeneric}
\pgfcirc@activate@bipole@simple{l}{short}
\pgfcirc@activate@bipole@simple{l}{open}


%% RESISTORS

% automatically switching path --- to be defined manually
\def\pgf@circ@resistor@path#1{\ifpgf@circuit@europeanresistor\pgf@circ@bipole@path{generic}{#1}\else\pgf@circ@bipole@path{resistor}{#1}\fi}
\pgfcirc@path@to@style{l}{resistor}{resistor}{}
\pgfcirc@node@to@style{l}{resistor}{american resistor}{}
\pgfcirc@node@to@style{l}{generic}{european resistor}{}
\pgfcirc@style@to@style{resistor}{R}

\def\pgf@circ@vresistor@path#1{\ifpgf@circuit@europeanresistor\pgf@circ@bipole@path{tgeneric}{#1}\else\pgf@circ@bipole@path{vresistor}{#1}\fi}
\pgfcirc@path@to@style{l}{vresistor}{variable resistor}{}
\pgfcirc@node@to@style{l}{vresistor}{variable american resistor}{}
\pgfcirc@node@to@style{l}{tgeneric}{variable european resistor}{}
\pgfcirc@style@to@style{variable resistor}{vR}

\def\pgf@circ@resistivesens@path#1{\ifpgf@circuit@europeanresistor\pgf@circ@bipole@path{thermistor}{#1}\else\pgf@circ@bipole@path{resistivesens}{#1}\fi}
\pgfcirc@path@to@style{l}{resistivesens}{resistive sensor}{}
\pgfcirc@node@to@style{l}{resistivesens}{american resistive sensor}{}
\pgfcirc@node@to@style{l}{thermistor}{european resistive sensor}{}
\pgfcirc@style@to@style{resistive sensor}{sR}

\def\pgf@circ@potentiometer@path#1{\ifpgf@circuit@europeanresistor\pgf@circ@bipole@path{genericpotentiometer}{#1}\else\pgf@circ@bipole@path{potentiometer}{#1}\fi}
\pgfcirc@path@to@style{l}{potentiometer}{potentiometer}{}
\pgfcirc@node@to@style{l}{potentiometer}{american potentiometer}{}
\pgfcirc@node@to@style{l}{genericpotentiometer}{european potentiometer}{}
\pgfcirc@style@to@style{potentiometer}{pR}

\pgfcirc@activate@bipole@simple{l}{thermistor}
\pgfcirc@style@to@style{thermistor}{thR}
\pgfcirc@activate@bipole{l}{thermistorptc}{thermistorptc}{thermistor ptc}
\pgfcirc@style@to@style{thermistor ptc}{thRp}
\pgfcirc@activate@bipole{l}{thermistorntc}{thermistorntc}{thermistor ntc}
\pgfcirc@style@to@style{thermistor ntc}{thRn}
\pgfcirc@activate@bipole@simple{l}{photoresistor}
\pgfcirc@style@to@style{photoresistor}{phR}
\pgfcirc@activate@bipole@simple{l}{varistor}
\pgfcirc@activate@bipole@simple{l}{memristor}
\pgfcirc@style@to@style{memristor}{Mr}

%% Capacitors

\pgfcirc@activate@bipole@simple{l}{capacitor}
\pgfcirc@style@to@style{capacitor}{C}
\pgfcirc@activate@bipole@simple{l}{ecapacitor}
\pgfcirc@style@to@style{ecapacitor}{eC}
\pgfcirc@style@to@style{ecapacitor}{elko}
\pgfcirc@activate@bipole{l}{polarcapacitor}{polarcapacitor}{polar capacitor}
%% polar capacitor is deprecated, use curved capacitor instead
\pgfcirc@style@to@style{polar capacitor}{pC}
\pgfcirc@activate@bipole{l}{ccapacitor}{ccapacitor}{curved capacitor}
\pgfcirc@style@to@style{curved capacitor}{cC}
\pgfcirc@activate@bipole{l}{vcapacitor}{vcapacitor}{variable capacitor}
\pgfcirc@style@to@style{variable capacitor}{vC}
\pgfcirc@activate@bipole@simple{l}{piezoelectric}
\pgfcirc@style@to@style{piezoelectric}{PZ}
\pgfcirc@activate@bipole{l}{capacitivesens}{capacitivesens}{capacitive sensor}
\pgfcirc@style@to@style{capacitive sensor}{sC}

%% Inductors
%% these are complex because of the three-way set
%% should be simplified
\def\pgf@circ@inductor@path#1{%
    \pgfextra{
        \edef\pgf@circ@temp{\ctikzvalof{inductor}}%
        \def\pgf@temp{european}%
    }
    \ifx\pgf@temp\pgf@circ@temp%
        \pgf@circ@europeaninductor@path{#1}%
    \else%
        \pgfextra{	\def\pgf@temp{cute} }%
        \ifx\pgf@temp\pgf@circ@temp%
            \pgf@circ@cuteinductor@path{#1}%
        \else%
            \pgf@circ@americaninductor@path{#1}%
        \fi%
    \fi%
}
\pgfcirc@path@to@style{l}{inductor}{inductor}{}
\pgfcirc@style@to@style{inductor}{L}
\pgfcirc@activate@bipole{l}{europeaninductor}{fullgeneric}{european inductor}
\pgfcirc@activate@bipole{l}{americaninductor}{americaninductor}{american inductor}
\pgfcirc@activate@bipole{l}{cuteinductor}{cuteinductor}{cute inductor}

\def\pgf@circ@vinductor@path#1{
    \pgfextra{
        \edef\pgf@circ@temp{\ctikzvalof{inductor}}%
        \def\pgf@temp{european}%
    }
    \ifx\pgf@temp\pgf@circ@temp%
        \pgf@circ@veuropeaninductor@path{#1}%
    \else%
        \pgfextra{	\def\pgf@temp{cute} }%
        \ifx\pgf@temp\pgf@circ@temp%
            \pgf@circ@vcuteinductor@path{#1}%
        \else%
            \pgf@circ@vamericaninductor@path{#1}%
        \fi%
    \fi%
}
\pgfcirc@path@to@style{l}{vinductor}{variable inductor}{}
\pgfcirc@style@to@style{variable inductor}{vL}
\pgfcirc@activate@bipole{l}{veuropeaninductor}{tfullgeneric}{variable european inductor}
\pgfcirc@activate@bipole{l}{vamericaninductor}{vamericaninductor}{variable american inductor}
\pgfcirc@activate@bipole{l}{vcuteinductor}{vcuteinductor}{variable cute inductor}

\def\pgf@circ@inductivesens@path#1{%
    \pgfextra{
        \edef\pgf@circ@temp{\ctikzvalof{inductor}}%
        \def\pgf@temp{european}%
    }
    \ifx\pgf@temp\pgf@circ@temp%
        \pgf@circ@europeaninductivesens@path{#1}%
    \else%
        \pgfextra{	\def\pgf@temp{cute} }%
        \ifx\pgf@temp\pgf@circ@temp%
            \pgf@circ@cuteinductivesens@path{#1}%
        \else%
            \pgf@circ@americaninductivesens@path{#1}%
        \fi%
    \fi%
}
\pgfcirc@path@to@style{l}{inductivesens}{inductive sensor}{}
\pgfcirc@style@to@style{inductive sensor}{sL}
\pgfcirc@activate@bipole{l}{europeaninductivesens}{sfullgeneric}{european inductive sensor}
\pgfcirc@activate@bipole{l}{americaninductivesens}{samericaninductor}{american inductive sensor}
\pgfcirc@activate@bipole{l}{cuteinductivesens}{scuteinductor}{cute inductive sensor}

\pgfcirc@activate@bipole{l}{cutechoke}{cutechoke}{cute choke}

%% Diodes

\def\pgfcirc@tmp@generatediodes#1#2{
    \pgfcirc@activate@bipole{l}{#1diode}{#1diode}{#1 diode}
    \pgfcirc@style@to@style{#1 diode}{D#2}
    \pgfcirc@activate@bipole{l}{#1zdiode}{#1zdiode}{#1 Zener diode}
    \pgfcirc@style@to@style{#1 Zener diode}{zD#2}
    \pgfcirc@activate@bipole{l}{#1zzdiode}{#1zzdiode}{#1 ZZener diode}
    \pgfcirc@style@to@style{#1 ZZener diode}{zzD#2}
    \pgfcirc@activate@bipole{l}{#1sdiode}{#1sdiode}{#1 Schottky diode}
    \pgfcirc@style@to@style{#1 Schottky diode}{sD#2}
    \pgfcirc@activate@bipole{l}{#1tdiode}{#1tdiode}{#1 tunnel diode}
    \pgfcirc@style@to@style{#1 tunnel diode}{tD#2}
    \pgfcirc@activate@bipole{l}{#1lediode}{#1lediode}{#1 led}
    \pgfcirc@style@to@style{#1 led}{leD#2}
    \pgfcirc@activate@bipole{l}{#1pdiode}{#1pdiode}{#1 photodiode}
    \pgfcirc@style@to@style{#1 photodiode}{pD#2}
    \pgfcirc@activate@bipole{l}{#1varcap}{#1varcap}{#1 varcap}
    \pgfcirc@style@to@style{#1 varcap}{VC#2}
    \pgfcirc@activate@bipole{l}{#1bidirectionaldiode}{#1bidirectionaldiode}{#1 bidirectionaldiode}
    \pgfcirc@style@to@style{#1 bidirectionaldiode}{biD#2}
    \pgfcirc@activate@bipole{l}{#1thyristor}{#1thyristor}{#1 thyristor}
    \pgfcirc@style@to@style{#1 thyristor}{Ty#2}
    \pgfcirc@activate@bipole{l}{#1triac}{#1triac}{#1 triac}
    \pgfcirc@style@to@style{#1 triac}{Tr#2}
}
\pgfcirc@tmp@generatediodes{full}{*}
\pgfcirc@tmp@generatediodes{empty}{o}
\def\pgfcirc@tmp@generatestrokeddiodes#1#2{
    \pgfcirc@node@to@style{l}{emptydiode}{#1 diode}{\circuitikzbasekey/bipole/is strokedsymbol=true}
    \pgfcirc@style@to@style{#1 diode}{D#2}
    \pgfcirc@node@to@style{l}{emptyzdiode}{#1 Zener diode}{\circuitikzbasekey/bipole/is strokedsymbol=true}
    \pgfcirc@style@to@style{#1 Zener diode}{zD#2}
    \pgfcirc@node@to@style{l}{emptyzzdiode}{#1 ZZener diode}{\circuitikzbasekey/bipole/is strokedsymbol=true}
    \pgfcirc@style@to@style{#1 ZZener diode}{zzD#2}
    \pgfcirc@node@to@style{l}{emptysdiode}{#1 Schottky diode}{\circuitikzbasekey/bipole/is strokedsymbol=true}
    \pgfcirc@style@to@style{#1 Schottky diode}{sD#2}
    \pgfcirc@node@to@style{l}{emptytdiode}{#1 tunnel diode}{\circuitikzbasekey/bipole/is strokedsymbol=true}
    \pgfcirc@style@to@style{#1 tunnel diode}{tD#2}
    \pgfcirc@node@to@style{l}{emptylediode}{#1 led}{\circuitikzbasekey/bipole/is strokedsymbol=true}
    \pgfcirc@style@to@style{#1 led}{leD#2}
    \pgfcirc@node@to@style{l}{emptypdiode}{#1 photodiode}{\circuitikzbasekey/bipole/is strokedsymbol=true}
    \pgfcirc@style@to@style{#1 photodiode}{pD#2}
    \pgfcirc@node@to@style{l}{emptyvarcap}{#1 varcap}{\circuitikzbasekey/bipole/is strokedsymbol=true}
    \pgfcirc@style@to@style{#1 varcap}{VC#2}
    \pgfcirc@node@to@style{l}{emptybidirectionaldiode}{#1 bidirectionaldiode}{\circuitikzbasekey/bipole/is strokedsymbol=true}
    \pgfcirc@style@to@style{#1 bidirectionaldiode}{biD#2}
    \pgfcirc@node@to@style{l}{emptythyristor}{#1 thyristor}{\circuitikzbasekey/bipole/is strokedsymbol=true}
    \pgfcirc@style@to@style{#1 thyristor}{Ty#2}
    \pgfcirc@node@to@style{l}{emptytriac}{#1 triac}{\circuitikzbasekey/bipole/is strokedsymbol=true}
    \pgfcirc@style@to@style{#1 triac}{Tr#2}
}
\pgfcirc@tmp@generatestrokeddiodes{stroke}{-}
\def\pgfcircdiodestylemacro{\ifpgf@circuit@strokediode stroke \else\ifpgf@circuit@fulldiode full \else empty \fi\fi}
% these are auto-switching styles
\pgfcirc@style@to@style{\pgfcircdiodestylemacro diode}{diode}
\pgfcirc@style@to@style{diode}{D}
\pgfcirc@style@to@style{\pgfcircdiodestylemacro Zener diode}{Zener diode}
\pgfcirc@style@to@style{Zener diode}{zD}
\pgfcirc@style@to@style{\pgfcircdiodestylemacro ZZener diode}{ZZener diode}
\pgfcirc@style@to@style{ZZener diode}{zzD}
\pgfcirc@style@to@style{\pgfcircdiodestylemacro Schottky diode}{Schottky diode}
\pgfcirc@style@to@style{Schottky diode}{sD}
\pgfcirc@style@to@style{\pgfcircdiodestylemacro tunnel diode}{tunnel diode}
\pgfcirc@style@to@style{tunnel diode}{tD}
\pgfcirc@style@to@style{\pgfcircdiodestylemacro led}{led}
\pgfcirc@style@to@style{led}{leD}
\pgfcirc@style@to@style{\pgfcircdiodestylemacro photodiode}{photodiode}
\pgfcirc@style@to@style{photodiode}{pD}
\pgfcirc@style@to@style{\pgfcircdiodestylemacro varcap}{varcap}
\pgfcirc@style@to@style{varcap}{VC}
\pgfcirc@style@to@style{\pgfcircdiodestylemacro bidirectionaldiode}{bidirectionaldiode}
\pgfcirc@style@to@style{bidirectionaldiode}{biD}
\pgfcirc@style@to@style{\pgfcircdiodestylemacro thyristor}{thyristor}
\pgfcirc@style@to@style{thyristor}{Ty}
\pgfcirc@style@to@style{\pgfcircdiodestylemacro triac}{triac}
\pgfcirc@style@to@style{triac}{Tr}

%% Batteries

\pgfcirc@activate@bipole@simple@opt{v}{battery}{\circuitikzbasekey/bipole/is voltage=true,
    \circuitikzbasekey/bipole/is voltageoutsideofsymbol=true}
\pgfcirc@activate@bipole@opt{v}{batteryone}{battery1}{battery1}{\circuitikzbasekey/bipole/is voltage=true,
    \circuitikzbasekey/bipole/is voltageoutsideofsymbol=true}
\pgfcirc@activate@bipole@opt{v}{batterytwo}{battery2}{battery2}{\circuitikzbasekey/bipole/is voltage=true,
    \circuitikzbasekey/bipole/is voltageoutsideofsymbol=true}

%% Sources
%% Sources: voltage

\pgfcirc@activate@bipole@opt{v}{vsource}{vsource}{european voltage source}{%
    \circuitikzbasekey/bipole/is voltage=true,
    \circuitikzbasekey/bipole/is voltageoutsideofsymbol=true}
\pgfcirc@activate@bipole@opt{v}{vsourceam}{vsourceAM}{american voltage source}{%
    \circuitikzbasekey/bipole/is voltage=true,
    \circuitikzbasekey/bipole/is voltageoutsideofsymbol=false}
\pgfcirc@style@to@style{\ifpgf@circuit@europeanvoltage european \else american \fi voltage source}{voltage source}
\pgfcirc@style@to@style{voltage source}{vsource}
\pgfcirc@style@to@style{voltage source}{V}

\pgfcirc@activate@bipole@opt{v}{cvsource}{cvsource}{european controlled voltage source}{%
    \circuitikzbasekey/bipole/is voltage=true,
    \circuitikzbasekey/bipole/is voltageoutsideofsymbol=true}
\pgfcirc@activate@bipole@opt{v}{cvsourceam}{cvsourceAM}{american controlled voltage source}{%
    \circuitikzbasekey/bipole/is voltage=true,
    \circuitikzbasekey/bipole/is voltageoutsideofsymbol=false}
\pgfcirc@style@to@style{\ifpgf@circuit@europeanvoltage european \else american \fi controlled voltage source}{controlled voltage source}
\pgfcirc@style@to@style{controlled voltage source}{cvsource}
\pgfcirc@style@to@style{controlled voltage source}{controlled vsource}
\pgfcirc@style@to@style{controlled voltage source}{cV}

\pgfcirc@activate@bipole@simple@opt{v}{esource}{%
    \circuitikzbasekey/bipole/is voltage=true,
    \circuitikzbasekey/bipole/is voltageoutsideofsymbol=true}

\pgfcirc@activate@bipole@opt{v}{ecsource}{ecsource}{empty controlled source}{%
    \circuitikzbasekey/bipole/is voltage=true,
    \circuitikzbasekey/bipole/is voltageoutsideofsymbol=true}
\pgfcirc@style@to@style{empty controlled source}{ecsource}

\pgfcirc@activate@bipole@opt{v}{vsourcesin}{vsourcesin}{sinusoidal voltage source}{%
    \circuitikzbasekey/bipole/is voltage=true,
    \circuitikzbasekey/bipole/is voltageoutsideofsymbol=true}
\pgfcirc@style@to@style{sinusoidal voltage source}{vsourcesin}
\pgfcirc@style@to@style{sinusoidal voltage source}{sV}

\pgfcirc@activate@bipole@opt{v}{cvsourcesin}{cvsourcesin}{controlled sinusoidal voltage source}{%
    \circuitikzbasekey/bipole/is voltage=true,
    \circuitikzbasekey/bipole/is voltageoutsideofsymbol=true}
\pgfcirc@style@to@style{controlled sinusoidal voltage source}{cvsourcesin}
\pgfcirc@style@to@style{controlled sinusoidal voltage source}{controlled vsourcesin}
\pgfcirc@style@to@style{controlled sinusoidal voltage source}{csV}

\pgfcirc@activate@bipole@opt{v}{vsourcesquare}{vsourcesquare}{square voltage source}{%
    \circuitikzbasekey/bipole/is voltage=true,
    \circuitikzbasekey/bipole/is voltageoutsideofsymbol=true}
\pgfcirc@style@to@style{square voltage source}{vsourcesquare}
\pgfcirc@style@to@style{square voltage source}{sqV}

\pgfcirc@activate@bipole@opt{v}{vsourcetri}{vsourcetri}{triangle voltage source}{%
    \circuitikzbasekey/bipole/is voltage=true,
    \circuitikzbasekey/bipole/is voltageoutsideofsymbol=true}
\pgfcirc@style@to@style{triangle voltage source}{vsourcetri}
\pgfcirc@style@to@style{triangle voltage source}{tV}

\pgfcirc@activate@bipole@simple@opt{v}{pvsource}{%
    \circuitikzbasekey/bipole/is voltage=true,
    \circuitikzbasekey/bipole/is voltageoutsideofsymbol=true}

\pgfcirc@activate@bipole@simple@opt{v}{dcvsource}{%
    \circuitikzbasekey/bipole/is voltage=true,
    \circuitikzbasekey/bipole/is voltageoutsideofsymbol=true}

\pgfcirc@activate@bipole@opt{v}{oosource}{oosource}{voosource}{%
    \circuitikzbasekey/bipole/is voltage=true,
    \circuitikzbasekey/bipole/is voltageoutsideofsymbol=true}

\pgfcirc@activate@bipole@simple@opt{v}{ooosource}{%
    \circuitikzbasekey/bipole/is voltage=true,
    \circuitikzbasekey/bipole/is voltageoutsideofsymbol=true}

\pgfcirc@activate@bipole@simple@opt{v}{oosourcetrans}{%
    \circuitikzbasekey/bipole/is voltage=true,
    \circuitikzbasekey/bipole/is voltageoutsideofsymbol=true}

\pgfcirc@activate@bipole@opt{v}{vsourceC}{vsourceC}{cute european voltage source}{%
    \circuitikzbasekey/bipole/is voltage=true,
    \circuitikzbasekey/bipole/is voltageoutsideofsymbol=true}
\pgfcirc@style@to@style{cute european voltage source}{vsourceC}
\pgfcirc@style@to@style{cute european voltage source}{ceV}

\pgfcirc@activate@bipole@opt{v}{cvsourceC}{cvsourceC}{cute european controlled voltage source}{%
    \circuitikzbasekey/bipole/is voltage=true,
    \circuitikzbasekey/bipole/is voltageoutsideofsymbol=true}
\pgfcirc@style@to@style{cute european controlled voltage source}{cvsourceC}
\pgfcirc@style@to@style{cute european controlled voltage source}{cceV}

\pgfcirc@activate@bipole@opt{v}{vsourceN}{vsourceN}{noise voltage source}{%
    \circuitikzbasekey/bipole/is voltage=true,
    \circuitikzbasekey/bipole/is voltageoutsideofsymbol=true}
\pgfcirc@style@to@style{noise voltage source}{vsourceN}
\pgfcirc@style@to@style{noise voltage source}{nV}

%% Sources: current

\pgfcirc@activate@bipole@opt{i}{isource}{isource}{european current source}{%
    \circuitikzbasekey/bipole/is current=true}
\pgfcirc@activate@bipole@opt{i}{isourceam}{isourceAM}{american current source}{%
    \circuitikzbasekey/bipole/is current=true}
\pgfcirc@style@to@style{\ifpgf@circuit@europeancurrent european \else american \fi current source}{current source}
\pgfcirc@style@to@style{current source}{isource}
\pgfcirc@style@to@style{current source}{I}

\pgfcirc@activate@bipole@opt{i}{cisource}{cisource}{european controlled current source}{%
    \circuitikzbasekey/bipole/is current=true}
\pgfcirc@activate@bipole@opt{i}{cisourceam}{cisourceAM}{american controlled current source}{%
    \circuitikzbasekey/bipole/is current=true}
\pgfcirc@style@to@style{\ifpgf@circuit@europeanvoltage european \else american \fi controlled current source}{controlled current source}
\pgfcirc@style@to@style{controlled current source}{cisource}
\pgfcirc@style@to@style{controlled current source}{controlled isource}
\pgfcirc@style@to@style{controlled current source}{cI}

\pgfcirc@activate@bipole@opt{i}{isourcesin}{isourcesin}{sinusoidal current source}{%
    \circuitikzbasekey/bipole/is current=true}
\pgfcirc@style@to@style{sinusoidal current source}{isourcesin}
\pgfcirc@style@to@style{sinusoidal current source}{sI}

\pgfcirc@activate@bipole@opt{i}{cisourcesin}{cisourcesin}{controlled sinusoidal current source}{%
    \circuitikzbasekey/bipole/is current=true}
\pgfcirc@style@to@style{controlled sinusoidal current source}{cisourcesin}
\pgfcirc@style@to@style{controlled sinusoidal current source}{controlled isourcesin}
\pgfcirc@style@to@style{controlled sinusoidal current source}{csI}

\pgfcirc@activate@bipole@simple@opt{i}{dcisource}{%
    \circuitikzbasekey/bipole/is current=true}

\pgfcirc@activate@bipole@opt{i}{oosource}{oosource}{ioosource}{%
    \circuitikzbasekey/bipole/is current=true}

\pgfcirc@activate@bipole@opt{i}{isourceC}{isourceC}{cute european current source}{%
    \circuitikzbasekey/bipole/is current=true}
\pgfcirc@style@to@style{cute european current source}{isourceC}
\pgfcirc@style@to@style{cute european current source}{ceI}

\pgfcirc@activate@bipole@opt{i}{cisourceC}{cisourceC}{cute european controlled current source}{%
    \circuitikzbasekey/bipole/is current=true}
\pgfcirc@style@to@style{cute european controlled current source}{cisourceC}
\pgfcirc@style@to@style{cute european controlled current source}{cceI}

\pgfcirc@activate@bipole@opt{i}{isourceN}{isourceN}{noise current source}{%
    \circuitikzbasekey/bipole/is current=true}
\pgfcirc@style@to@style{noise current source}{isourceN}
\pgfcirc@style@to@style{noise current source}{nI}

% build alias with voltage and current directions (legacy)

\def\pgf@temp#1{
    \pgfcirc@style@to@style@label{voltage source}{V#1}{v#1}
    \pgfcirc@style@to@style@label{controlled voltage source}{cV#1}{v#1}
    \pgfcirc@style@to@style@label{sinusoidal voltage source}{sV#1}{v#1}
    \pgfcirc@style@to@style@label{controlled sinusoidal voltage source}{csV#1}{v#1}
}
\pgf@temp{_>} \pgf@temp{_<} \pgf@temp{^>} \pgf@temp{^<}
\pgf@temp{>} \pgf@temp{<} \pgf@temp{^} \pgf@temp{_}
\def\pgf@temp#1{
    \pgfcirc@style@to@style@label{current source}{I#1}{i#1}
    \pgfcirc@style@to@style@label{controlled current source}{cI#1}{i#1}
    \pgfcirc@style@to@style@label{sinusoidal current source}{sI#1}{i#1}
    \pgfcirc@style@to@style@label{controlled sinusoidal current source}{csI#1}{i#1}
}
\pgf@temp{_>} \pgf@temp{_<} \pgf@temp{^>} \pgf@temp{^<}
\pgf@temp{>_} \pgf@temp{<_} \pgf@temp{>^} \pgf@temp{<^}
\pgf@temp{>} \pgf@temp{<} \pgf@temp{^} \pgf@temp{_}

% Instruments

\pgfcirc@activate@bipole@simple{l}{ammeter}
\pgfcirc@activate@bipole@simple{l}{ohmmeter}
\pgfcirc@activate@bipole@simple{l}{voltmeter}
\pgfcirc@activate@bipole@simple{l}{oscope}
\pgfcirc@activate@bipole@simple{l}{rmeter}
\pgfcirc@activate@bipole@simple{l}{rmeterwa}
\pgfcirc@activate@bipole@simple{l}{smeter}
\pgfcirc@activate@bipole@simple{l}{iloop}
% \pgfcirc@activate@bipole@simple{l}{iloop2} that was wrong
\pgfcirc@activate@bipole{l}{ilooptwo}{iloop2}{iloop2}
\pgfcirc@activate@bipole@simple{l}{qvprobe}
\pgfcirc@activate@bipole@simple{l}{qiprobe}
\pgfcirc@activate@bipole@simple{l}{qpprobe}

%% Mechanical

\pgfcirc@activate@bipole@simple{l}{spring}
\pgfcirc@activate@bipole@simple{l}{inerter}
\pgfcirc@activate@bipole@simple{l}{mass}
\pgfcirc@activate@bipole@simple{l}{damper}
\pgfcirc@activate@bipole@simple{l}{viscoe}

%% Miscellaneous

\pgfcirc@activate@bipole@simple{l}{lamp}
\pgfcirc@activate@bipole@simple{l}{bulb}
\pgfcirc@activate@bipole@simple{l}{squid}
\pgfcirc@activate@bipole@simple{l}{barrier}
\pgfcirc@activate@bipole@simple{l}{openbarrier}
\pgfcirc@activate@bipole@simple{l}{thermocouple}
\pgfcirc@activate@bipole@simple{l}{fuse}
\pgfcirc@activate@bipole{l}{afuse}{afuse}{asymmetric fuse}
\pgfcirc@style@to@style{asymmetric fuse}{afuse}
\def\pgf@circ@gfsurgearrester@path#1{\ifpgf@circuit@europeangfsurgearrester\pgf@circ@europeangfsurgearrester@path{#1}\else\pgf@circ@americangfsurgearrester@path{#1}\fi}
\pgfcirc@activate@bipole{l}{europeangfsurgearrester}{european gas filled surge arrester}{european gas filled surge arrester}
\pgfcirc@activate@bipole{l}{americangfsurgearrester}{american gas filled surge arrester}{american gas filled surge arrester}
\pgfcirc@path@to@style{l}{gfsurgearrester}{gas filled surge arrester}{}
\pgfcirc@path@to@style{l}{gfsurgearrester}{gf surge arrester}{}
\pgfcirc@activate@bipole@simple{l}{mic}
\pgfcirc@activate@bipole@simple{l}{loudspeaker}

%% Switches and buttons

\pgfcirc@activate@bipole{l}{cspst}{cspst}{closing switch}
\pgfcirc@style@to@style{closing switch}{switch}
\pgfcirc@style@to@style{closing switch}{cspst}
\pgfcirc@style@to@style{switch}{spst}
\pgfcirc@activate@bipole{l}{ospst}{ospst}{opening switch}
\pgfcirc@style@to@style{opening switch}{ospst}

\pgfcirc@activate@bipole@simple{l}{nos}
\pgfcirc@style@to@style{nos}{normal open switch}
\pgfcirc@activate@bipole@simple{l}{ncs}
\pgfcirc@style@to@style{ncs}{normal closed switch}

\pgfcirc@activate@bipole{l}{pushbutton}{pushbutton}{push button}
\pgfcirc@style@to@style{push button}{nopb}
\pgfcirc@style@to@style{push button}{normally open push button}
\pgfcirc@activate@bipole{l}{ncpushbutton}{ncpushbutton}{ncpb}
\pgfcirc@style@to@style{ncpb}{normally closed push button}
\pgfcirc@activate@bipole{l}{pushbuttonc}{pushbuttonc}{nopbc}
\pgfcirc@style@to@style{nopbc}{normally open push button closed}
\pgfcirc@activate@bipole{l}{ncpushbuttono}{ncpushbuttono}{ncpbo}
\pgfcirc@style@to@style{ncpbo}{normally closed push button open}

\pgfcirc@activate@bipole{l}{toggleswitch}{toggleswitch}{toggle switch}
\pgfcirc@activate@bipole@simple{l}{reed}

\pgfcirc@activate@bipole{l}{cuteclosedswitch}{cuteclosedswitch}{cute closed switch}
\pgfcirc@style@to@style{cute closed switch}{ccsw}
\pgfcirc@activate@bipole{l}{cuteopenswitch}{cuteopenswitch}{cute open switch}
\pgfcirc@style@to@style{cute open switch}{cosw}
\pgfcirc@activate@bipole{l}{cuteclosingswitch}{cuteclosingswitch}{cute closing switch}
\pgfcirc@style@to@style{cute closing switch}{ccgsw}
\pgfcirc@activate@bipole{l}{cuteopeningswitch}{cuteopeningswitch}{cute opening switch}
\pgfcirc@style@to@style{cute opening switch}{cogsw}
%% Path definitions



\def\pgf@circ@tline@path#1{\pgf@circ@bipole@path{tline}{#1}}
\def\pgf@circ@mstline@path#1{\pgf@circ@bipole@path{mstline}{#1}}


\def\pgf@circ@twoport@path#1{\pgf@circ@bipole@path{twoport}{#1}}
\def\pgf@circ@twoportsplit@path#1{\pgf@circ@bipole@path{twoportsplit}{#1}}
\def\pgf@circ@vco@path#1{\pgf@circ@bipole@path{vco}{#1}}
\def\pgf@circ@bandpass@path#1{\pgf@circ@bipole@path{bandpass}{#1}}
\def\pgf@circ@bandstop@path#1{\pgf@circ@bipole@path{bandstop}{#1}}
\def\pgf@circ@highpass@path#1{\pgf@circ@bipole@path{highpass}{#1}}
\def\pgf@circ@lowpass@path#1{\pgf@circ@bipole@path{lowpass}{#1}}
\def\pgf@circ@allpass@path#1{\pgf@circ@bipole@path{allpass}{#1}}
\def\pgf@circ@adc@path#1{\pgf@circ@bipole@path{adc}{#1}}
\def\pgf@circ@dac@path#1{\pgf@circ@bipole@path{dac}{#1}}
\def\pgf@circ@dsp@path#1{\pgf@circ@bipole@path{dsp}{#1}}
\def\pgf@circ@fft@path#1{\pgf@circ@bipole@path{fft}{#1}}
\def\pgf@circ@amp@path#1{\pgf@circ@bipole@path{amp}{#1}}
\def\pgf@circ@vamp@path#1{\pgf@circ@bipole@path{vamp}{#1}}
\def\pgf@circ@piattenuator@path#1{\pgf@circ@bipole@path{piattenuator}{#1}}
\def\pgf@circ@vpiattenuator@path#1{\pgf@circ@bipole@path{vpiattenuator}{#1}}
\def\pgf@circ@tattenuator@path#1{\pgf@circ@bipole@path{tattenuator}{#1}}
\def\pgf@circ@vtattenuator@path#1{\pgf@circ@bipole@path{vtattenuator}{#1}}
\def\pgf@circ@phaseshifter@path#1{\pgf@circ@bipole@path{phaseshifter}{#1}}
\def\pgf@circ@vphaseshifter@path#1{\pgf@circ@bipole@path{vphaseshifter}{#1}}
\def\pgf@circ@detector@path#1{\pgf@circ@bipole@path{detector}{#1}}
%
\def\pgf@circ@sacdc@path#1{\pgf@circ@bipole@path{sacdc}{#1}}
\def\pgf@circ@sdcac@path#1{\pgf@circ@bipole@path{sdcac}{#1}}
\def\pgf@circ@tacdc@path#1{\pgf@circ@bipole@path{tacdc}{#1}}
\def\pgf@circ@tdcac@path#1{\pgf@circ@bipole@path{tdcac}{#1}}


\compattikzset{tline/.style = {\circuitikzbasekey, /tikz/to path=\pgf@circ@tline@path, l=#1}}
\compattikzset{transmission line/.style = {tline = #1}}
\compattikzset{TL/.style = {tline = #1}}
\compattikzset{mstline/.style = {\circuitikzbasekey, /tikz/to path=\pgf@circ@mstline@path, l=#1}}


\compattikzset{twoport/.style = {\circuitikzbasekey, /tikz/to path=\pgf@circ@twoport@path}}
\compattikzset{twoportsplit/.style = {\circuitikzbasekey, /tikz/to path=\pgf@circ@twoportsplit@path}}
\compattikzset{vco/.style = {\circuitikzbasekey, /tikz/to path=\pgf@circ@vco@path}}
\compattikzset{bandpass/.style = {\circuitikzbasekey, /tikz/to path=\pgf@circ@bandpass@path}}
\compattikzset{bandstop/.style = {\circuitikzbasekey, /tikz/to path=\pgf@circ@bandstop@path}}
\compattikzset{highpass/.style = {\circuitikzbasekey, /tikz/to path=\pgf@circ@highpass@path}}
\compattikzset{lowpass/.style = {\circuitikzbasekey, /tikz/to path=\pgf@circ@lowpass@path}}
\compattikzset{allpass/.style = {\circuitikzbasekey, /tikz/to path=\pgf@circ@allpass@path}}
\compattikzset{adc/.style = {\circuitikzbasekey, /tikz/to path=\pgf@circ@adc@path}}
\compattikzset{dac/.style = {\circuitikzbasekey, /tikz/to path=\pgf@circ@dac@path}}
\compattikzset{dsp/.style = {\circuitikzbasekey, /tikz/to path=\pgf@circ@dsp@path}}
\compattikzset{fft/.style = {\circuitikzbasekey, /tikz/to path=\pgf@circ@fft@path}}
\compattikzset{amp/.style = {\circuitikzbasekey, /tikz/to path=\pgf@circ@amp@path}}
\compattikzset{vamp/.style = {\circuitikzbasekey, /tikz/to path=\pgf@circ@vamp@path}}
\compattikzset{piattenuator/.style = {\circuitikzbasekey, /tikz/to path=\pgf@circ@piattenuator@path}}
\compattikzset{vpiattenuator/.style = {\circuitikzbasekey, /tikz/to path=\pgf@circ@vpiattenuator@path}}
\compattikzset{tattenuator/.style = {\circuitikzbasekey, /tikz/to path=\pgf@circ@tattenuator@path}}
\compattikzset{vtattenuator/.style = {\circuitikzbasekey, /tikz/to path=\pgf@circ@vtattenuator@path}}
\compattikzset{phaseshifter/.style = {\circuitikzbasekey, /tikz/to path=\pgf@circ@phaseshifter@path}}
\compattikzset{vphaseshifter/.style = {\circuitikzbasekey, /tikz/to path=\pgf@circ@vphaseshifter@path}}
\compattikzset{detector/.style = {\circuitikzbasekey, /tikz/to path=\pgf@circ@detector@path}}
%
\compattikzset{sacdc/.style = {\circuitikzbasekey, /tikz/to path=\pgf@circ@sacdc@path, l=#1}}
\compattikzset{sdcac/.style = {\circuitikzbasekey, /tikz/to path=\pgf@circ@sdcac@path, l=#1}}
\compattikzset{tacdc/.style = {\circuitikzbasekey, /tikz/to path=\pgf@circ@tacdc@path, l=#1}}
\compattikzset{tdcac/.style = {\circuitikzbasekey, /tikz/to path=\pgf@circ@tdcac@path, l=#1}}


%% Define Shortcuts





\compattikzset{vdd/.style = {\comnpatname vcc = #1}}
\compattikzset{vss/.style = {\comnpatname vee = #1}}

% activate the to-style crossing
\def\pgf@circ@crossing@path#1{\pgf@circ@bipole@path{crossing}{#1}}
\compattikzset{crossing/.style = {\circuitikzbasekey, /tikz/to path=\pgf@circ@crossing@path, l=#1}}
\compattikzset{xing/.style= {\comnpatname crossing= #1}}


% multiwire(s)
\def\pgf@circ@bmultiwire@path#1{\pgf@circ@bipole@path{bmultiwire}{#1}}
\compattikzset{bmultiwire/.style = {\circuitikzbasekey,
/tikz/to path=\pgf@circ@bmultiwire@path, l=#1}}
\def\pgf@circ@multiwire@path#1{\pgf@circ@bipole@path{multiwire}{#1}}
\compattikzset{multiwire/.style = {\circuitikzbasekey,
/tikz/to path=\pgf@circ@multiwire@path, l=#1}}
\def\pgf@circ@tmultiwire@path#1{\pgf@circ@bipole@path{tmultiwire}{#1}}
\compattikzset{tmultiwire/.style = {\circuitikzbasekey,
/tikz/to path=\pgf@circ@tmultiwire@path, l=#1}}


% Transistor like bipoles

\def\pgf@circ@trans@path#1#2{
    \pgfextra{
        \edef\pgf@temp{\ctikzvalof{bipole/name}}
        \def\pgf@circ@temp{#2}
        \ifx\pgf@temp\pgf@circ@temp % if it has not a name
            \pgfmathrandominteger{\pgf@circ@rand}{1000}{9999}
            \ctikzset{bipole/name = trans\pgf@circ@rand} % create it
        \fi
    }
    \ifpgf@circuit@bipole@inverted
        (\tikztostart) node[coordinate] (\ctikzvalof{bipole/name}end) {}
        (\tikztotarget) node[coordinate] (\ctikzvalof{bipole/name}start) {}
    \else
        (\tikztostart) node[coordinate] (\ctikzvalof{bipole/name}start) {}
        (\tikztotarget) node[coordinate] (\ctikzvalof{bipole/name}end) {}
    \fi
    \pgfextra{
        \pgfmathanglebetweenpoints{\pgfpointanchor{\ctikzvalof{bipole/name}start}{center}}
        {\pgfpointanchor{\ctikzvalof{bipole/name}end}{center}}
        \pgfmathadd{\pgfmathresult}{-90}
        \pgfmathround{\pgfmathresult}
        \edef\pgf@circ@direction{\pgfmathresult}
    }
    ($(\tikztostart) ! .5 ! (\tikztotarget)$)
    node[#1, /tikz/rotate=\pgf@circ@direction, xscale=\ctikzvalof{mirror value}]
    (\ctikzvalof{bipole/name}) {}
    node {\ctikzvalof{bipole/label/name}}
    \ifcsname pgf@anchor@#1@pathstart\endcsname%if special path-anchors are defined, use them!
        (\ctikzvalof{bipole/name}start.center) --(\ctikzvalof{bipole/name}.pathstart)
        (\ctikzvalof{bipole/name}.pathend)  -- (\ctikzvalof{bipole/name}end.center)
    \else
        (\ctikzvalof{bipole/name}start.center) --(\ctikzvalof{bipole/name}.left)
        (\ctikzvalof{bipole/name}.right)  -- (\ctikzvalof{bipole/name}end.center)
    \fi
    \drawpoles
    \pgfextra{
        \pgfcircresetpath
    }
    (\tikztotarget) 	\tikztonodes  % e si continua
}


\def\pgf@circ@definetranspath#1{
	\compattikzset{T#1/.style = {\circuitikzbasekey, /tikz/to path=\pgf@circ@trans@path{#1}{}, l=##1}}
}

\pgf@circ@definetranspath{elmech}
\pgf@circ@definetranspath{nmos}
\pgf@circ@definetranspath{pmos}
\pgf@circ@definetranspath{nmosd}
\pgf@circ@definetranspath{pmosd}
\pgf@circ@definetranspath{hemt}
\pgf@circ@definetranspath{npn}
\pgf@circ@definetranspath{pnp}
\pgf@circ@definetranspath{nfet}
\pgf@circ@definetranspath{nigfete}
\pgf@circ@definetranspath{nigfetd}
\pgf@circ@definetranspath{nigfetebulk}
\pgf@circ@definetranspath{pfet}
\pgf@circ@definetranspath{pigfete}
\pgf@circ@definetranspath{pigfetd}
\pgf@circ@definetranspath{pigfetebulk}
\pgf@circ@definetranspath{njfet}
\pgf@circ@definetranspath{pjfet}
\pgf@circ@definetranspath{pigbt}
\pgf@circ@definetranspath{nigbt}
\pgf@circ@definetranspath{Lpigbt}
\pgf@circ@definetranspath{Lnigbt}
%
% Path-style logical ports
%
% create path-style element for one input --- one output logical ports
%
\def\pgfcirc@port@node@to@path#1#2{%
    %
    % add a logic port path style component --- we need to suppress leads
    % and use the correct center
    %
    \pgfcirc@node@to@path{#1}{#2}{/tikz/no leads, \circuitikzbasekey/logic ports origin=center}%
}
\pgfcirc@port@node@to@path{not port}{inline not}
\pgfcirc@port@node@to@path{buffer port}{inline buffer}
\pgfcirc@port@node@to@path{schmitt port}{inline schmitt}
\pgfcirc@port@node@to@path{invschmitt port}{inline invschmitt}

\pgfcirc@port@node@to@path{tgate}{inline tgate}
\pgfcirc@port@node@to@path{double tgate}{inline double tgate}

